\documentclass[11pt]{article}

    \usepackage[breakable]{tcolorbox}
    \usepackage{parskip} % Stop auto-indenting (to mimic markdown behaviour)
    
    \usepackage{iftex}
    \ifPDFTeX
    	\usepackage[T1]{fontenc}
    	\usepackage{mathpazo}
    \else
    	\usepackage{fontspec}
    \fi

    % Basic figure setup, for now with no caption control since it's done
    % automatically by Pandoc (which extracts ![](path) syntax from Markdown).
    \usepackage{graphicx}
    % Maintain compatibility with old templates. Remove in nbconvert 6.0
    \let\Oldincludegraphics\includegraphics
    % Ensure that by default, figures have no caption (until we provide a
    % proper Figure object with a Caption API and a way to capture that
    % in the conversion process - todo).
    \usepackage{caption}
    \DeclareCaptionFormat{nocaption}{}
    \captionsetup{format=nocaption,aboveskip=0pt,belowskip=0pt}

    \usepackage[Export]{adjustbox} % Used to constrain images to a maximum size
    \adjustboxset{max size={0.9\linewidth}{0.9\paperheight}}
    \usepackage{float}
    \floatplacement{figure}{H} % forces figures to be placed at the correct location
    \usepackage{xcolor} % Allow colors to be defined
    \usepackage{enumerate} % Needed for markdown enumerations to work
    \usepackage{geometry} % Used to adjust the document margins
    \usepackage{amsmath} % Equations
    \usepackage{amssymb} % Equations
    \usepackage{textcomp} % defines textquotesingle
    % Hack from http://tex.stackexchange.com/a/47451/13684:
    \AtBeginDocument{%
        \def\PYZsq{\textquotesingle}% Upright quotes in Pygmentized code
    }
    \usepackage{upquote} % Upright quotes for verbatim code
    \usepackage{eurosym} % defines \euro
    \usepackage[mathletters]{ucs} % Extended unicode (utf-8) support
    \usepackage{fancyvrb} % verbatim replacement that allows latex
    \usepackage{grffile} % extends the file name processing of package graphics 
                         % to support a larger range
    \makeatletter % fix for grffile with XeLaTeX
    \def\Gread@@xetex#1{%
      \IfFileExists{"\Gin@base".bb}%
      {\Gread@eps{\Gin@base.bb}}%
      {\Gread@@xetex@aux#1}%
    }
    \makeatother

    % The hyperref package gives us a pdf with properly built
    % internal navigation ('pdf bookmarks' for the table of contents,
    % internal cross-reference links, web links for URLs, etc.)
    \usepackage{hyperref}
    % The default LaTeX title has an obnoxious amount of whitespace. By default,
    % titling removes some of it. It also provides customization options.
    \usepackage{titling}
    \usepackage{longtable} % longtable support required by pandoc >1.10
    \usepackage{booktabs}  % table support for pandoc > 1.12.2
    \usepackage[inline]{enumitem} % IRkernel/repr support (it uses the enumerate* environment)
    \usepackage[normalem]{ulem} % ulem is needed to support strikethroughs (\sout)
                                % normalem makes italics be italics, not underlines
    \usepackage{mathrsfs}
    

    
    % Colors for the hyperref package
    \definecolor{urlcolor}{rgb}{0,.145,.698}
    \definecolor{linkcolor}{rgb}{.71,0.21,0.01}
    \definecolor{citecolor}{rgb}{.12,.54,.11}

    % ANSI colors
    \definecolor{ansi-black}{HTML}{3E424D}
    \definecolor{ansi-black-intense}{HTML}{282C36}
    \definecolor{ansi-red}{HTML}{E75C58}
    \definecolor{ansi-red-intense}{HTML}{B22B31}
    \definecolor{ansi-green}{HTML}{00A250}
    \definecolor{ansi-green-intense}{HTML}{007427}
    \definecolor{ansi-yellow}{HTML}{DDB62B}
    \definecolor{ansi-yellow-intense}{HTML}{B27D12}
    \definecolor{ansi-blue}{HTML}{208FFB}
    \definecolor{ansi-blue-intense}{HTML}{0065CA}
    \definecolor{ansi-magenta}{HTML}{D160C4}
    \definecolor{ansi-magenta-intense}{HTML}{A03196}
    \definecolor{ansi-cyan}{HTML}{60C6C8}
    \definecolor{ansi-cyan-intense}{HTML}{258F8F}
    \definecolor{ansi-white}{HTML}{C5C1B4}
    \definecolor{ansi-white-intense}{HTML}{A1A6B2}
    \definecolor{ansi-default-inverse-fg}{HTML}{FFFFFF}
    \definecolor{ansi-default-inverse-bg}{HTML}{000000}

    % commands and environments needed by pandoc snippets
    % extracted from the output of `pandoc -s`
    \providecommand{\tightlist}{%
      \setlength{\itemsep}{0pt}\setlength{\parskip}{0pt}}
    \DefineVerbatimEnvironment{Highlighting}{Verbatim}{commandchars=\\\{\}}
    % Add ',fontsize=\small' for more characters per line
    \newenvironment{Shaded}{}{}
    \newcommand{\KeywordTok}[1]{\textcolor[rgb]{0.00,0.44,0.13}{\textbf{{#1}}}}
    \newcommand{\DataTypeTok}[1]{\textcolor[rgb]{0.56,0.13,0.00}{{#1}}}
    \newcommand{\DecValTok}[1]{\textcolor[rgb]{0.25,0.63,0.44}{{#1}}}
    \newcommand{\BaseNTok}[1]{\textcolor[rgb]{0.25,0.63,0.44}{{#1}}}
    \newcommand{\FloatTok}[1]{\textcolor[rgb]{0.25,0.63,0.44}{{#1}}}
    \newcommand{\CharTok}[1]{\textcolor[rgb]{0.25,0.44,0.63}{{#1}}}
    \newcommand{\StringTok}[1]{\textcolor[rgb]{0.25,0.44,0.63}{{#1}}}
    \newcommand{\CommentTok}[1]{\textcolor[rgb]{0.38,0.63,0.69}{\textit{{#1}}}}
    \newcommand{\OtherTok}[1]{\textcolor[rgb]{0.00,0.44,0.13}{{#1}}}
    \newcommand{\AlertTok}[1]{\textcolor[rgb]{1.00,0.00,0.00}{\textbf{{#1}}}}
    \newcommand{\FunctionTok}[1]{\textcolor[rgb]{0.02,0.16,0.49}{{#1}}}
    \newcommand{\RegionMarkerTok}[1]{{#1}}
    \newcommand{\ErrorTok}[1]{\textcolor[rgb]{1.00,0.00,0.00}{\textbf{{#1}}}}
    \newcommand{\NormalTok}[1]{{#1}}
    
    % Additional commands for more recent versions of Pandoc
    \newcommand{\ConstantTok}[1]{\textcolor[rgb]{0.53,0.00,0.00}{{#1}}}
    \newcommand{\SpecialCharTok}[1]{\textcolor[rgb]{0.25,0.44,0.63}{{#1}}}
    \newcommand{\VerbatimStringTok}[1]{\textcolor[rgb]{0.25,0.44,0.63}{{#1}}}
    \newcommand{\SpecialStringTok}[1]{\textcolor[rgb]{0.73,0.40,0.53}{{#1}}}
    \newcommand{\ImportTok}[1]{{#1}}
    \newcommand{\DocumentationTok}[1]{\textcolor[rgb]{0.73,0.13,0.13}{\textit{{#1}}}}
    \newcommand{\AnnotationTok}[1]{\textcolor[rgb]{0.38,0.63,0.69}{\textbf{\textit{{#1}}}}}
    \newcommand{\CommentVarTok}[1]{\textcolor[rgb]{0.38,0.63,0.69}{\textbf{\textit{{#1}}}}}
    \newcommand{\VariableTok}[1]{\textcolor[rgb]{0.10,0.09,0.49}{{#1}}}
    \newcommand{\ControlFlowTok}[1]{\textcolor[rgb]{0.00,0.44,0.13}{\textbf{{#1}}}}
    \newcommand{\OperatorTok}[1]{\textcolor[rgb]{0.40,0.40,0.40}{{#1}}}
    \newcommand{\BuiltInTok}[1]{{#1}}
    \newcommand{\ExtensionTok}[1]{{#1}}
    \newcommand{\PreprocessorTok}[1]{\textcolor[rgb]{0.74,0.48,0.00}{{#1}}}
    \newcommand{\AttributeTok}[1]{\textcolor[rgb]{0.49,0.56,0.16}{{#1}}}
    \newcommand{\InformationTok}[1]{\textcolor[rgb]{0.38,0.63,0.69}{\textbf{\textit{{#1}}}}}
    \newcommand{\WarningTok}[1]{\textcolor[rgb]{0.38,0.63,0.69}{\textbf{\textit{{#1}}}}}
    
    
    % Define a nice break command that doesn't care if a line doesn't already
    % exist.
    \def\br{\hspace*{\fill} \\* }
    % Math Jax compatibility definitions
    \def\gt{>}
    \def\lt{<}
    \let\Oldtex\TeX
    \let\Oldlatex\LaTeX
    \renewcommand{\TeX}{\textrm{\Oldtex}}
    \renewcommand{\LaTeX}{\textrm{\Oldlatex}}
    % Document parameters
    % Document title
    \title{How to invest on CaC40}
   \author{Miradain Atontsa ( miradain.atontsan@gmail.com ) }    
    
    
    
    
% Pygments definitions
\makeatletter
\def\PY@reset{\let\PY@it=\relax \let\PY@bf=\relax%
    \let\PY@ul=\relax \let\PY@tc=\relax%
    \let\PY@bc=\relax \let\PY@ff=\relax}
\def\PY@tok#1{\csname PY@tok@#1\endcsname}
\def\PY@toks#1+{\ifx\relax#1\empty\else%
    \PY@tok{#1}\expandafter\PY@toks\fi}
\def\PY@do#1{\PY@bc{\PY@tc{\PY@ul{%
    \PY@it{\PY@bf{\PY@ff{#1}}}}}}}
\def\PY#1#2{\PY@reset\PY@toks#1+\relax+\PY@do{#2}}

\expandafter\def\csname PY@tok@w\endcsname{\def\PY@tc##1{\textcolor[rgb]{0.73,0.73,0.73}{##1}}}
\expandafter\def\csname PY@tok@c\endcsname{\let\PY@it=\textit\def\PY@tc##1{\textcolor[rgb]{0.25,0.50,0.50}{##1}}}
\expandafter\def\csname PY@tok@cp\endcsname{\def\PY@tc##1{\textcolor[rgb]{0.74,0.48,0.00}{##1}}}
\expandafter\def\csname PY@tok@k\endcsname{\let\PY@bf=\textbf\def\PY@tc##1{\textcolor[rgb]{0.00,0.50,0.00}{##1}}}
\expandafter\def\csname PY@tok@kp\endcsname{\def\PY@tc##1{\textcolor[rgb]{0.00,0.50,0.00}{##1}}}
\expandafter\def\csname PY@tok@kt\endcsname{\def\PY@tc##1{\textcolor[rgb]{0.69,0.00,0.25}{##1}}}
\expandafter\def\csname PY@tok@o\endcsname{\def\PY@tc##1{\textcolor[rgb]{0.40,0.40,0.40}{##1}}}
\expandafter\def\csname PY@tok@ow\endcsname{\let\PY@bf=\textbf\def\PY@tc##1{\textcolor[rgb]{0.67,0.13,1.00}{##1}}}
\expandafter\def\csname PY@tok@nb\endcsname{\def\PY@tc##1{\textcolor[rgb]{0.00,0.50,0.00}{##1}}}
\expandafter\def\csname PY@tok@nf\endcsname{\def\PY@tc##1{\textcolor[rgb]{0.00,0.00,1.00}{##1}}}
\expandafter\def\csname PY@tok@nc\endcsname{\let\PY@bf=\textbf\def\PY@tc##1{\textcolor[rgb]{0.00,0.00,1.00}{##1}}}
\expandafter\def\csname PY@tok@nn\endcsname{\let\PY@bf=\textbf\def\PY@tc##1{\textcolor[rgb]{0.00,0.00,1.00}{##1}}}
\expandafter\def\csname PY@tok@ne\endcsname{\let\PY@bf=\textbf\def\PY@tc##1{\textcolor[rgb]{0.82,0.25,0.23}{##1}}}
\expandafter\def\csname PY@tok@nv\endcsname{\def\PY@tc##1{\textcolor[rgb]{0.10,0.09,0.49}{##1}}}
\expandafter\def\csname PY@tok@no\endcsname{\def\PY@tc##1{\textcolor[rgb]{0.53,0.00,0.00}{##1}}}
\expandafter\def\csname PY@tok@nl\endcsname{\def\PY@tc##1{\textcolor[rgb]{0.63,0.63,0.00}{##1}}}
\expandafter\def\csname PY@tok@ni\endcsname{\let\PY@bf=\textbf\def\PY@tc##1{\textcolor[rgb]{0.60,0.60,0.60}{##1}}}
\expandafter\def\csname PY@tok@na\endcsname{\def\PY@tc##1{\textcolor[rgb]{0.49,0.56,0.16}{##1}}}
\expandafter\def\csname PY@tok@nt\endcsname{\let\PY@bf=\textbf\def\PY@tc##1{\textcolor[rgb]{0.00,0.50,0.00}{##1}}}
\expandafter\def\csname PY@tok@nd\endcsname{\def\PY@tc##1{\textcolor[rgb]{0.67,0.13,1.00}{##1}}}
\expandafter\def\csname PY@tok@s\endcsname{\def\PY@tc##1{\textcolor[rgb]{0.73,0.13,0.13}{##1}}}
\expandafter\def\csname PY@tok@sd\endcsname{\let\PY@it=\textit\def\PY@tc##1{\textcolor[rgb]{0.73,0.13,0.13}{##1}}}
\expandafter\def\csname PY@tok@si\endcsname{\let\PY@bf=\textbf\def\PY@tc##1{\textcolor[rgb]{0.73,0.40,0.53}{##1}}}
\expandafter\def\csname PY@tok@se\endcsname{\let\PY@bf=\textbf\def\PY@tc##1{\textcolor[rgb]{0.73,0.40,0.13}{##1}}}
\expandafter\def\csname PY@tok@sr\endcsname{\def\PY@tc##1{\textcolor[rgb]{0.73,0.40,0.53}{##1}}}
\expandafter\def\csname PY@tok@ss\endcsname{\def\PY@tc##1{\textcolor[rgb]{0.10,0.09,0.49}{##1}}}
\expandafter\def\csname PY@tok@sx\endcsname{\def\PY@tc##1{\textcolor[rgb]{0.00,0.50,0.00}{##1}}}
\expandafter\def\csname PY@tok@m\endcsname{\def\PY@tc##1{\textcolor[rgb]{0.40,0.40,0.40}{##1}}}
\expandafter\def\csname PY@tok@gh\endcsname{\let\PY@bf=\textbf\def\PY@tc##1{\textcolor[rgb]{0.00,0.00,0.50}{##1}}}
\expandafter\def\csname PY@tok@gu\endcsname{\let\PY@bf=\textbf\def\PY@tc##1{\textcolor[rgb]{0.50,0.00,0.50}{##1}}}
\expandafter\def\csname PY@tok@gd\endcsname{\def\PY@tc##1{\textcolor[rgb]{0.63,0.00,0.00}{##1}}}
\expandafter\def\csname PY@tok@gi\endcsname{\def\PY@tc##1{\textcolor[rgb]{0.00,0.63,0.00}{##1}}}
\expandafter\def\csname PY@tok@gr\endcsname{\def\PY@tc##1{\textcolor[rgb]{1.00,0.00,0.00}{##1}}}
\expandafter\def\csname PY@tok@ge\endcsname{\let\PY@it=\textit}
\expandafter\def\csname PY@tok@gs\endcsname{\let\PY@bf=\textbf}
\expandafter\def\csname PY@tok@gp\endcsname{\let\PY@bf=\textbf\def\PY@tc##1{\textcolor[rgb]{0.00,0.00,0.50}{##1}}}
\expandafter\def\csname PY@tok@go\endcsname{\def\PY@tc##1{\textcolor[rgb]{0.53,0.53,0.53}{##1}}}
\expandafter\def\csname PY@tok@gt\endcsname{\def\PY@tc##1{\textcolor[rgb]{0.00,0.27,0.87}{##1}}}
\expandafter\def\csname PY@tok@err\endcsname{\def\PY@bc##1{\setlength{\fboxsep}{0pt}\fcolorbox[rgb]{1.00,0.00,0.00}{1,1,1}{\strut ##1}}}
\expandafter\def\csname PY@tok@kc\endcsname{\let\PY@bf=\textbf\def\PY@tc##1{\textcolor[rgb]{0.00,0.50,0.00}{##1}}}
\expandafter\def\csname PY@tok@kd\endcsname{\let\PY@bf=\textbf\def\PY@tc##1{\textcolor[rgb]{0.00,0.50,0.00}{##1}}}
\expandafter\def\csname PY@tok@kn\endcsname{\let\PY@bf=\textbf\def\PY@tc##1{\textcolor[rgb]{0.00,0.50,0.00}{##1}}}
\expandafter\def\csname PY@tok@kr\endcsname{\let\PY@bf=\textbf\def\PY@tc##1{\textcolor[rgb]{0.00,0.50,0.00}{##1}}}
\expandafter\def\csname PY@tok@bp\endcsname{\def\PY@tc##1{\textcolor[rgb]{0.00,0.50,0.00}{##1}}}
\expandafter\def\csname PY@tok@fm\endcsname{\def\PY@tc##1{\textcolor[rgb]{0.00,0.00,1.00}{##1}}}
\expandafter\def\csname PY@tok@vc\endcsname{\def\PY@tc##1{\textcolor[rgb]{0.10,0.09,0.49}{##1}}}
\expandafter\def\csname PY@tok@vg\endcsname{\def\PY@tc##1{\textcolor[rgb]{0.10,0.09,0.49}{##1}}}
\expandafter\def\csname PY@tok@vi\endcsname{\def\PY@tc##1{\textcolor[rgb]{0.10,0.09,0.49}{##1}}}
\expandafter\def\csname PY@tok@vm\endcsname{\def\PY@tc##1{\textcolor[rgb]{0.10,0.09,0.49}{##1}}}
\expandafter\def\csname PY@tok@sa\endcsname{\def\PY@tc##1{\textcolor[rgb]{0.73,0.13,0.13}{##1}}}
\expandafter\def\csname PY@tok@sb\endcsname{\def\PY@tc##1{\textcolor[rgb]{0.73,0.13,0.13}{##1}}}
\expandafter\def\csname PY@tok@sc\endcsname{\def\PY@tc##1{\textcolor[rgb]{0.73,0.13,0.13}{##1}}}
\expandafter\def\csname PY@tok@dl\endcsname{\def\PY@tc##1{\textcolor[rgb]{0.73,0.13,0.13}{##1}}}
\expandafter\def\csname PY@tok@s2\endcsname{\def\PY@tc##1{\textcolor[rgb]{0.73,0.13,0.13}{##1}}}
\expandafter\def\csname PY@tok@sh\endcsname{\def\PY@tc##1{\textcolor[rgb]{0.73,0.13,0.13}{##1}}}
\expandafter\def\csname PY@tok@s1\endcsname{\def\PY@tc##1{\textcolor[rgb]{0.73,0.13,0.13}{##1}}}
\expandafter\def\csname PY@tok@mb\endcsname{\def\PY@tc##1{\textcolor[rgb]{0.40,0.40,0.40}{##1}}}
\expandafter\def\csname PY@tok@mf\endcsname{\def\PY@tc##1{\textcolor[rgb]{0.40,0.40,0.40}{##1}}}
\expandafter\def\csname PY@tok@mh\endcsname{\def\PY@tc##1{\textcolor[rgb]{0.40,0.40,0.40}{##1}}}
\expandafter\def\csname PY@tok@mi\endcsname{\def\PY@tc##1{\textcolor[rgb]{0.40,0.40,0.40}{##1}}}
\expandafter\def\csname PY@tok@il\endcsname{\def\PY@tc##1{\textcolor[rgb]{0.40,0.40,0.40}{##1}}}
\expandafter\def\csname PY@tok@mo\endcsname{\def\PY@tc##1{\textcolor[rgb]{0.40,0.40,0.40}{##1}}}
\expandafter\def\csname PY@tok@ch\endcsname{\let\PY@it=\textit\def\PY@tc##1{\textcolor[rgb]{0.25,0.50,0.50}{##1}}}
\expandafter\def\csname PY@tok@cm\endcsname{\let\PY@it=\textit\def\PY@tc##1{\textcolor[rgb]{0.25,0.50,0.50}{##1}}}
\expandafter\def\csname PY@tok@cpf\endcsname{\let\PY@it=\textit\def\PY@tc##1{\textcolor[rgb]{0.25,0.50,0.50}{##1}}}
\expandafter\def\csname PY@tok@c1\endcsname{\let\PY@it=\textit\def\PY@tc##1{\textcolor[rgb]{0.25,0.50,0.50}{##1}}}
\expandafter\def\csname PY@tok@cs\endcsname{\let\PY@it=\textit\def\PY@tc##1{\textcolor[rgb]{0.25,0.50,0.50}{##1}}}

\def\PYZbs{\char`\\}
\def\PYZus{\char`\_}
\def\PYZob{\char`\{}
\def\PYZcb{\char`\}}
\def\PYZca{\char`\^}
\def\PYZam{\char`\&}
\def\PYZlt{\char`\<}
\def\PYZgt{\char`\>}
\def\PYZsh{\char`\#}
\def\PYZpc{\char`\%}
\def\PYZdl{\char`\$}
\def\PYZhy{\char`\-}
\def\PYZsq{\char`\'}
\def\PYZdq{\char`\"}
\def\PYZti{\char`\~}
% for compatibility with earlier versions
\def\PYZat{@}
\def\PYZlb{[}
\def\PYZrb{]}
\makeatother


    % For linebreaks inside Verbatim environment from package fancyvrb. 
    \makeatletter
        \newbox\Wrappedcontinuationbox 
        \newbox\Wrappedvisiblespacebox 
        \newcommand*\Wrappedvisiblespace {\textcolor{red}{\textvisiblespace}} 
        \newcommand*\Wrappedcontinuationsymbol {\textcolor{red}{\llap{\tiny$\m@th\hookrightarrow$}}} 
        \newcommand*\Wrappedcontinuationindent {3ex } 
        \newcommand*\Wrappedafterbreak {\kern\Wrappedcontinuationindent\copy\Wrappedcontinuationbox} 
        % Take advantage of the already applied Pygments mark-up to insert 
        % potential linebreaks for TeX processing. 
        %        {, <, #, %, $, ' and ": go to next line. 
        %        _, }, ^, &, >, - and ~: stay at end of broken line. 
        % Use of \textquotesingle for straight quote. 
        \newcommand*\Wrappedbreaksatspecials {% 
            \def\PYGZus{\discretionary{\char`\_}{\Wrappedafterbreak}{\char`\_}}% 
            \def\PYGZob{\discretionary{}{\Wrappedafterbreak\char`\{}{\char`\{}}% 
            \def\PYGZcb{\discretionary{\char`\}}{\Wrappedafterbreak}{\char`\}}}% 
            \def\PYGZca{\discretionary{\char`\^}{\Wrappedafterbreak}{\char`\^}}% 
            \def\PYGZam{\discretionary{\char`\&}{\Wrappedafterbreak}{\char`\&}}% 
            \def\PYGZlt{\discretionary{}{\Wrappedafterbreak\char`\<}{\char`\<}}% 
            \def\PYGZgt{\discretionary{\char`\>}{\Wrappedafterbreak}{\char`\>}}% 
            \def\PYGZsh{\discretionary{}{\Wrappedafterbreak\char`\#}{\char`\#}}% 
            \def\PYGZpc{\discretionary{}{\Wrappedafterbreak\char`\%}{\char`\%}}% 
            \def\PYGZdl{\discretionary{}{\Wrappedafterbreak\char`\$}{\char`\$}}% 
            \def\PYGZhy{\discretionary{\char`\-}{\Wrappedafterbreak}{\char`\-}}% 
            \def\PYGZsq{\discretionary{}{\Wrappedafterbreak\textquotesingle}{\textquotesingle}}% 
            \def\PYGZdq{\discretionary{}{\Wrappedafterbreak\char`\"}{\char`\"}}% 
            \def\PYGZti{\discretionary{\char`\~}{\Wrappedafterbreak}{\char`\~}}% 
        } 
        % Some characters . , ; ? ! / are not pygmentized. 
        % This macro makes them "active" and they will insert potential linebreaks 
        \newcommand*\Wrappedbreaksatpunct {% 
            \lccode`\~`\.\lowercase{\def~}{\discretionary{\hbox{\char`\.}}{\Wrappedafterbreak}{\hbox{\char`\.}}}% 
            \lccode`\~`\,\lowercase{\def~}{\discretionary{\hbox{\char`\,}}{\Wrappedafterbreak}{\hbox{\char`\,}}}% 
            \lccode`\~`\;\lowercase{\def~}{\discretionary{\hbox{\char`\;}}{\Wrappedafterbreak}{\hbox{\char`\;}}}% 
            \lccode`\~`\:\lowercase{\def~}{\discretionary{\hbox{\char`\:}}{\Wrappedafterbreak}{\hbox{\char`\:}}}% 
            \lccode`\~`\?\lowercase{\def~}{\discretionary{\hbox{\char`\?}}{\Wrappedafterbreak}{\hbox{\char`\?}}}% 
            \lccode`\~`\!\lowercase{\def~}{\discretionary{\hbox{\char`\!}}{\Wrappedafterbreak}{\hbox{\char`\!}}}% 
            \lccode`\~`\/\lowercase{\def~}{\discretionary{\hbox{\char`\/}}{\Wrappedafterbreak}{\hbox{\char`\/}}}% 
            \catcode`\.\active
            \catcode`\,\active 
            \catcode`\;\active
            \catcode`\:\active
            \catcode`\?\active
            \catcode`\!\active
            \catcode`\/\active 
            \lccode`\~`\~ 	
        }
    \makeatother

    \let\OriginalVerbatim=\Verbatim
    \makeatletter
    \renewcommand{\Verbatim}[1][1]{%
        %\parskip\z@skip
        \sbox\Wrappedcontinuationbox {\Wrappedcontinuationsymbol}%
        \sbox\Wrappedvisiblespacebox {\FV@SetupFont\Wrappedvisiblespace}%
        \def\FancyVerbFormatLine ##1{\hsize\linewidth
            \vtop{\raggedright\hyphenpenalty\z@\exhyphenpenalty\z@
                \doublehyphendemerits\z@\finalhyphendemerits\z@
                \strut ##1\strut}%
        }%
        % If the linebreak is at a space, the latter will be displayed as visible
        % space at end of first line, and a continuation symbol starts next line.
        % Stretch/shrink are however usually zero for typewriter font.
        \def\FV@Space {%
            \nobreak\hskip\z@ plus\fontdimen3\font minus\fontdimen4\font
            \discretionary{\copy\Wrappedvisiblespacebox}{\Wrappedafterbreak}
            {\kern\fontdimen2\font}%
        }%
        
        % Allow breaks at special characters using \PYG... macros.
        \Wrappedbreaksatspecials
        % Breaks at punctuation characters . , ; ? ! and / need catcode=\active 	
        \OriginalVerbatim[#1,codes*=\Wrappedbreaksatpunct]%
    }
    \makeatother

    % Exact colors from NB
    \definecolor{incolor}{HTML}{303F9F}
    \definecolor{outcolor}{HTML}{D84315}
    \definecolor{cellborder}{HTML}{CFCFCF}
    \definecolor{cellbackground}{HTML}{F7F7F7}
    
    % prompt
    \makeatletter
    \newcommand{\boxspacing}{\kern\kvtcb@left@rule\kern\kvtcb@boxsep}
    \makeatother
    \newcommand{\prompt}[4]{
        \ttfamily\llap{{\color{#2}[#3]:\hspace{3pt}#4}}\vspace{-\baselineskip}
    }
    

    
    % Prevent overflowing lines due to hard-to-break entities
    \sloppy 
    % Setup hyperref package
    \hypersetup{
      breaklinks=true,  % so long urls are correctly broken across lines
      colorlinks=true,
      urlcolor=urlcolor,
      linkcolor=linkcolor,
      citecolor=citecolor,
      }
    % Slightly bigger margins than the latex defaults
    
    \geometry{verbose,tmargin=1in,bmargin=1in,lmargin=1in,rmargin=1in}
    
    

\begin{document}
    
    \maketitle
    
    

    
    If you wanted to invest on CaC40, what should be your strategy? Should
you invest on the benchmark index or rather selecting a couple of
companies from the index?

It is always difficult to build a diversified portfolio that beat the
benchmark index. In the literature, the theory named ``Enhanced
Indexing'' is meant to address this problem. In this paper, we present
an alternative to traditional methods using topological data analysis
(TDA). More precisely, we study the CaC40 index and build two portfolios
that beat this benchmark index according to the profile risk appetite.


    \begin{tcolorbox}[breakable, size=fbox, boxrule=1pt, pad at break*=1mm,colback=cellbackground, colframe=cellborder]
%\prompt{In}{incolor}{61}{\boxspacing}
\begin{Verbatim}[commandchars=\\\{\}]
\PY{k+kn}{import} \PY{n+nn}{pandas\PYZus{}datareader} \PY{k}{as} \PY{n+nn}{pdr}
\PY{k+kn}{import} \PY{n+nn}{datetime} 
\PY{k+kn}{import} \PY{n+nn}{numpy} \PY{k}{as} \PY{n+nn}{np}
\PY{k+kn}{import} \PY{n+nn}{pandas} \PY{k}{as} \PY{n+nn}{pd}
\PY{k+kn}{import} \PY{n+nn}{matplotlib}\PY{n+nn}{.}\PY{n+nn}{pyplot} \PY{k}{as} \PY{n+nn}{plt} 
\PY{k+kn}{import} \PY{n+nn}{statsmodels}\PY{n+nn}{.}\PY{n+nn}{api} \PY{k}{as} \PY{n+nn}{sm}
\PY{k+kn}{from} \PY{n+nn}{yahoo\PYZus{}fin} \PY{k+kn}{import} \PY{n}{stock\PYZus{}info} \PY{k}{as} \PY{n}{si} 
\end{Verbatim}
\end{tcolorbox}



    \hypertarget{loading-the-data-from-yahoo-finance}{%
\section{Loading the data from Yahoo
Finance}\label{loading-the-data-from-yahoo-finance}}

\begin{itemize}
\tightlist
\item
  Extracting data function
\end{itemize}

    \begin{tcolorbox}[breakable, size=fbox, boxrule=1pt, pad at break*=1mm,colback=cellbackground, colframe=cellborder]
%\prompt{In}{incolor}{62}{\boxspacing}
\begin{Verbatim}[commandchars=\\\{\}]
\PY{k}{def} \PY{n+nf}{get}\PY{p}{(}\PY{n}{tickers}\PY{p}{,} \PY{n}{startdate}\PY{p}{,} \PY{n}{enddate}\PY{p}{)}\PY{p}{:}
  \PY{k}{def} \PY{n+nf}{data}\PY{p}{(}\PY{n}{ticker}\PY{p}{)}\PY{p}{:}
    \PY{k}{return} \PY{p}{(}\PY{n}{pdr}\PY{o}{.}\PY{n}{get\PYZus{}data\PYZus{}yahoo}\PY{p}{(}\PY{n}{ticker}\PY{p}{,} \PY{n}{start}\PY{o}{=}\PY{n}{startdate}\PY{p}{,} \PY{n}{end}\PY{o}{=}\PY{n}{enddate}\PY{p}{)}\PY{p}{)}
  \PY{n}{datas} \PY{o}{=} \PY{n+nb}{map} \PY{p}{(}\PY{n}{data}\PY{p}{,} \PY{n}{tickers}\PY{p}{)}
  \PY{k}{return}\PY{p}{(}\PY{n}{pd}\PY{o}{.}\PY{n}{concat}\PY{p}{(}\PY{n}{datas}\PY{p}{,} \PY{n}{keys}\PY{o}{=}\PY{n}{tickers}\PY{p}{,} \PY{n}{names}\PY{o}{=}\PY{p}{[}\PY{l+s+s1}{\PYZsq{}}\PY{l+s+s1}{Ticker}\PY{l+s+s1}{\PYZsq{}}\PY{p}{,} \PY{l+s+s1}{\PYZsq{}}\PY{l+s+s1}{Date}\PY{l+s+s1}{\PYZsq{}}\PY{p}{]}\PY{p}{)}\PY{p}{)}
\end{Verbatim}
\end{tcolorbox}

    \begin{itemize}
\tightlist
\item
  The 30 most important stocks in the CaC40 index
\end{itemize}

    \begin{tcolorbox}[breakable, size=fbox, boxrule=1pt, pad at break*=1mm,colback=cellbackground, colframe=cellborder]
%\prompt{In}{incolor}{58}{\boxspacing}
\begin{Verbatim}[commandchars=\\\{\}]
\PY{n}{tickers}\PY{o}{=}\PY{p}{[}\PY{l+s+s1}{\PYZsq{}}\PY{l+s+s1}{SGO.PA}\PY{l+s+s1}{\PYZsq{}}\PY{p}{,}\PY{l+s+s1}{\PYZsq{}}\PY{l+s+s1}{RI.PA}\PY{l+s+s1}{\PYZsq{}}\PY{p}{,}  \PY{l+s+s1}{\PYZsq{}}\PY{l+s+s1}{LR.PA}\PY{l+s+s1}{\PYZsq{}}\PY{p}{,} \PY{l+s+s1}{\PYZsq{}}\PY{l+s+s1}{EN.PA}\PY{l+s+s1}{\PYZsq{}}\PY{p}{,} \PY{l+s+s1}{\PYZsq{}}\PY{l+s+s1}{AI.PA}\PY{l+s+s1}{\PYZsq{}}\PY{p}{,} \PY{l+s+s1}{\PYZsq{}}\PY{l+s+s1}{HO.PA}\PY{l+s+s1}{\PYZsq{}}\PY{p}{,}\PY{l+s+s1}{\PYZsq{}}\PY{l+s+s1}{ML.PA}\PY{l+s+s1}{\PYZsq{}}\PY{p}{,}\PY{l+s+s1}{\PYZsq{}}\PY{l+s+s1}{VIV.PA}\PY{l+s+s1}{\PYZsq{}}\PY{p}{,}\PY{l+s+s1}{\PYZsq{}}\PY{l+s+s1}{AIR.PA}\PY{l+s+s1}{\PYZsq{}}\PY{p}{,}\PY{l+s+s1}{\PYZsq{}}\PY{l+s+s1}{ACA.PA}\PY{l+s+s1}{\PYZsq{}}\PY{p}{,}\PY{l+s+s1}{\PYZsq{}}\PY{l+s+s1}{ENGI.PA}\PY{l+s+s1}{\PYZsq{}}\PY{p}{,}
        \PY{l+s+s1}{\PYZsq{}}\PY{l+s+s1}{ATO.PA}\PY{l+s+s1}{\PYZsq{}}\PY{p}{,}\PY{l+s+s1}{\PYZsq{}}\PY{l+s+s1}{BN.PA}\PY{l+s+s1}{\PYZsq{}}\PY{p}{,}\PY{l+s+s1}{\PYZsq{}}\PY{l+s+s1}{SW.PA}\PY{l+s+s1}{\PYZsq{}}\PY{p}{,}\PY{l+s+s1}{\PYZsq{}}\PY{l+s+s1}{VIE.PA}\PY{l+s+s1}{\PYZsq{}}\PY{p}{,}\PY{l+s+s1}{\PYZsq{}}\PY{l+s+s1}{BNP.PA}\PY{l+s+s1}{\PYZsq{}}\PY{p}{,}\PY{l+s+s1}{\PYZsq{}}\PY{l+s+s1}{SAN.PA}\PY{l+s+s1}{\PYZsq{}}\PY{p}{,}\PY{l+s+s1}{\PYZsq{}}\PY{l+s+s1}{UG.PA}\PY{l+s+s1}{\PYZsq{}}\PY{p}{,}\PY{l+s+s1}{\PYZsq{}}\PY{l+s+s1}{KER.PA}\PY{l+s+s1}{\PYZsq{}}\PY{p}{,}\PY{l+s+s1}{\PYZsq{}}\PY{l+s+s1}{MC.PA}\PY{l+s+s1}{\PYZsq{}}\PY{p}{,}\PY{l+s+s1}{\PYZsq{}}\PY{l+s+s1}{ORA.PA}\PY{l+s+s1}{\PYZsq{}}\PY{p}{,}\PY{l+s+s1}{\PYZsq{}}\PY{l+s+s1}{GLE.PA}\PY{l+s+s1}{\PYZsq{}}\PY{p}{,}\PY{l+s+s1}{\PYZsq{}}\PY{l+s+s1}{SU.PA}\PY{l+s+s1}{\PYZsq{}}\PY{p}{,}\PY{l+s+s1}{\PYZsq{}}\PY{l+s+s1}{DG.PA}\PY{l+s+s1}{\PYZsq{}}\PY{p}{,}
        \PY{l+s+s1}{\PYZsq{}}\PY{l+s+s1}{OR.PA}\PY{l+s+s1}{\PYZsq{}}\PY{p}{,}\PY{l+s+s1}{\PYZsq{}}\PY{l+s+s1}{CA.PA}\PY{l+s+s1}{\PYZsq{}}\PY{p}{,}\PY{l+s+s1}{\PYZsq{}}\PY{l+s+s1}{CAP.PA}\PY{l+s+s1}{\PYZsq{}}\PY{p}{,}\PY{l+s+s1}{\PYZsq{}}\PY{l+s+s1}{AC.PA}\PY{l+s+s1}{\PYZsq{}}\PY{p}{,}\PY{l+s+s1}{\PYZsq{}}\PY{l+s+s1}{FP.PA}\PY{l+s+s1}{\PYZsq{}}\PY{p}{,} \PY{l+s+s1}{\PYZsq{}}\PY{l+s+s1}{TEP.PA}\PY{l+s+s1}{\PYZsq{}}\PY{p}{]}
\end{Verbatim}
\end{tcolorbox}

    \begin{itemize}
\tightlist
\item
  Extracting the stock data from Yahoo Finance
\end{itemize}

    \begin{tcolorbox}[breakable, size=fbox, boxrule=1pt, pad at break*=1mm,colback=cellbackground, colframe=cellborder]
%\prompt{In}{incolor}{63}{\boxspacing}
\begin{Verbatim}[commandchars=\\\{\}]
\PY{c+c1}{\PYZsh{} Get the stock data}
\PY{n}{data} \PY{o}{=} \PY{n}{get}\PY{p}{(}\PY{n}{tickers}\PY{p}{,} \PY{n}{datetime}\PY{o}{.}\PY{n}{datetime}\PY{p}{(}\PY{l+m+mi}{2005}\PY{p}{,} \PY{l+m+mi}{1}\PY{p}{,} \PY{l+m+mi}{1}\PY{p}{)}\PY{p}{,} \PY{n}{datetime}\PY{o}{.}\PY{n}{datetime}\PY{p}{(}\PY{l+m+mi}{2020}\PY{p}{,} \PY{l+m+mi}{6}\PY{p}{,} \PY{l+m+mi}{30}\PY{p}{)}\PY{p}{)}
\PY{c+c1}{\PYZsh{} Isolate the `Adj Close` values and transform the DataFrame}
\PY{n}{portfolio} \PY{o}{=} \PY{n}{data}\PY{p}{[}\PY{p}{[}\PY{l+s+s1}{\PYZsq{}}\PY{l+s+s1}{Adj Close}\PY{l+s+s1}{\PYZsq{}}\PY{p}{]}\PY{p}{]}\PY{o}{.}\PY{n}{reset\PYZus{}index}\PY{p}{(}\PY{p}{)}\PY{o}{.}\PY{n}{pivot}\PY{p}{(}\PY{l+s+s1}{\PYZsq{}}\PY{l+s+s1}{Date}\PY{l+s+s1}{\PYZsq{}}\PY{p}{,} \PY{l+s+s1}{\PYZsq{}}\PY{l+s+s1}{Ticker}\PY{l+s+s1}{\PYZsq{}}\PY{p}{,} \PY{l+s+s1}{\PYZsq{}}\PY{l+s+s1}{Adj Close}\PY{l+s+s1}{\PYZsq{}}\PY{p}{)}
\end{Verbatim}
\end{tcolorbox}

    \begin{itemize}
\tightlist
\item
  Extracting the CaC40 index data from Yahoo Finance
\end{itemize}

    \begin{tcolorbox}[breakable, size=fbox, boxrule=1pt, pad at break*=1mm,colback=cellbackground, colframe=cellborder]
%\prompt{In}{incolor}{64}{\boxspacing}
\begin{Verbatim}[commandchars=\\\{\}]
\PY{n}{ticker\PYZus{}cac40}\PY{o}{=} \PY{p}{[}\PY{l+s+s1}{\PYZsq{}}\PY{l+s+s1}{\PYZca{}FCHI}\PY{l+s+s1}{\PYZsq{}}\PY{p}{]}
\PY{n}{portfolio\PYZus{}I} \PY{o}{=} \PY{n}{get}\PY{p}{(}\PY{n}{ticker\PYZus{}cac40}\PY{p}{,} \PY{n}{datetime}\PY{o}{.}\PY{n}{datetime}\PY{p}{(}\PY{l+m+mi}{2005}\PY{p}{,} \PY{l+m+mi}{1}\PY{p}{,} \PY{l+m+mi}{1}\PY{p}{)}\PY{p}{,} \PY{n}{datetime}\PY{o}{.}\PY{n}{datetime}\PY{p}{(}\PY{l+m+mi}{2020}\PY{p}{,} \PY{l+m+mi}{6}\PY{p}{,} \PY{l+m+mi}{30}\PY{p}{)}\PY{p}{)}
\PY{n}{index\PYZus{}I} \PY{o}{=} \PY{n}{portfolio\PYZus{}I}\PY{p}{[}\PY{p}{[}\PY{l+s+s1}{\PYZsq{}}\PY{l+s+s1}{Adj Close}\PY{l+s+s1}{\PYZsq{}}\PY{p}{]}\PY{p}{]}\PY{o}{.}\PY{n}{reset\PYZus{}index}\PY{p}{(}\PY{p}{)}\PY{o}{.}\PY{n}{pivot}\PY{p}{(}\PY{l+s+s1}{\PYZsq{}}\PY{l+s+s1}{Date}\PY{l+s+s1}{\PYZsq{}}\PY{p}{,} \PY{l+s+s1}{\PYZsq{}}\PY{l+s+s1}{Ticker}\PY{l+s+s1}{\PYZsq{}}\PY{p}{,} \PY{l+s+s1}{\PYZsq{}}\PY{l+s+s1}{Adj Close}\PY{l+s+s1}{\PYZsq{}}\PY{p}{)}
\end{Verbatim}
\end{tcolorbox}

    \begin{itemize}
\tightlist
\item
  Computing the (log) returns in the portfolio including the benchmark
  index
\end{itemize}

    \begin{tcolorbox}[breakable, size=fbox, boxrule=1pt, pad at break*=1mm,colback=cellbackground, colframe=cellborder]
%\prompt{In}{incolor}{66}{\boxspacing}
\begin{Verbatim}[commandchars=\\\{\}]
\PY{k}{def} \PY{n+nf}{asset\PYZus{}log\PYZus{}returns}\PY{p}{(}\PY{n}{index}\PY{p}{,} \PY{n}{portfolio}\PY{p}{)}\PY{p}{:}
    \PY{n}{New\PYZus{}portfolio}\PY{o}{=} \PY{n}{portfolio}
    \PY{n}{New\PYZus{}portfolio}\PY{p}{[}\PY{n}{index}\PY{p}{]}\PY{o}{=}\PY{n}{index\PYZus{}I}\PY{p}{[}\PY{n}{index}\PY{p}{]}
    \PY{n}{asset\PYZus{}returns}\PY{o}{=} \PY{n}{np}\PY{o}{.}\PY{n}{log}\PY{p}{(}\PY{l+m+mi}{1} \PY{o}{+} \PY{n}{New\PYZus{}portfolio}\PY{o}{.}\PY{n}{pct\PYZus{}change}\PY{p}{(}\PY{p}{)}\PY{p}{)}\PY{o}{.}\PY{n}{dropna}\PY{p}{(}\PY{p}{)}
    \PY{k}{return} \PY{n}{asset\PYZus{}returns}

\PY{n}{asset\PYZus{}returns}\PY{o}{=} \PY{n}{asset\PYZus{}log\PYZus{}returns}\PY{p}{(}\PY{l+s+s1}{\PYZsq{}}\PY{l+s+s1}{\PYZca{}FCHI}\PY{l+s+s1}{\PYZsq{}}\PY{p}{,} \PY{n}{portfolio}\PY{p}{)}
\end{Verbatim}
\end{tcolorbox}



    \hypertarget{topological-data-analysis-tda}{%
\section{Topological data analysis
(TDA)}\label{topological-data-analysis-tda}}

Topological data analysis is a method invented to analyze data with a
strong emphasis on its shape. This theory has sometimes been shown to be
more robust than traditional methods and in some contexts it is used in
combination with statistical data analysis methods. In addition, it
offers excellent visualization tools for larger dimension data points.

Topological data analysis studies the shape of the data from its point
cloud (see the plot below).

\[  \adjustimage{max size={0.9\linewidth}{0.9\paperheight}}{TDA_pipeline.jpg} \]

Namely, from a point cloud data, we build a filtration by thickening the
points to balls with larger and larger radius. As the radius of balls
encreases, circles are created and then filled out. We records the date
of birth and death of each ones. This is the topological information,
also called ``Persistent diagram'' that we will next try to search the
meaning. However this valuable information is not yet appropriate for a
Data Science job. We have to construct a real data feature out of
persistent diagram. This process is called ``Vectorization''. One way of
vectorizing persistent diagrams (as shown in the plot, bottom right)
consists of gluing together isosceles right triangles built on each
features. This is called ``Persistent Landscape''.

In general topological data analysis has the following pipeline:

\[ \{ \text{Point cloud data }\} \Longrightarrow  \{ \text{Persistent Diagrams }\}  \Longrightarrow  \{ \text{Vectorization }\}  \Longrightarrow  \{ \text{ Data Science }\}\]

    \hypertarget{implementation-of-tda-in-finance}{%
\section{Implementation of TDA in
finance}\label{implementation-of-tda-in-finance}}

In finance, TDA has a longer pipeline:

\[ \{ \text{Data processing }\} \Longrightarrow  \{ \text{Point cloud data }\} \Longrightarrow  \{ \text{Persistent Diagrams }\}  \Longrightarrow  \{ \text{Vectorization }\}   \Downarrow \]

\ \ \ \ \ \ \ \ \ \ \ \  \ \ \ \ \ \ \ \ \ \ \ \   \ \ \ \ \ \ \ \ \ \ \ \  \[ \Longrightarrow  \{ \text{ Portfolio constructions}\}\]


We will now briefly explain each step.

\begin{itemize}
\tightlist
\item
  Data processing = Taken's embedding.
\end{itemize}

Financial stock data are essentially one dimensional ( time series )
data. Thus we don't have yet a point cloud for TDA analysis. We will use
the Taken's embedding theorem to transform the stock data into a point
cloud in \(\mathbb{R}^{d}\) for some integer \(d;\)

\begin{itemize}
\tightlist
\item
  Persistent Diagrams = Use python's TDA library.
\end{itemize}

We will use the TDA-python library ``Ripser'' to compute the persistent
diagram.

\begin{itemize}
\tightlist
\item
  Vectorization = Persistent landscape + \(L^p\) norm series.
\end{itemize}

Normally a single point cloud data produces a single persistent diagram
and then a single persistent landscape. We will use the sliding window
technic to get a series of persistent landscapes. Moreover, knowing that
the persistent landscapes form a subset of the banach space
\(L^p(\mathbb{N} \times \mathbb{R}),\) we can compute the \(L^p\) norm
for each persistent landscape. Therefore we get a series of norm values
which is now the topological feature to consider for data analysis.

\begin{itemize}
\tightlist
\item
  Financial analysis = Use the topological feature to analyze the stocks
  and build portfolios.
\end{itemize}

Roughly speaking we will only keep assets having highest or lowest
\(L^p\) norm.

    \begin{tcolorbox}[breakable, size=fbox, boxrule=1pt, pad at break*=1mm,colback=cellbackground, colframe=cellborder]
%\prompt{In}{incolor}{71}{\boxspacing}
\begin{Verbatim}[commandchars=\\\{\}]
\PY{c+c1}{\PYZsh{} Loading the library for topological data analysis}
\PY{k+kn}{import} \PY{n+nn}{numpy} \PY{k}{as} \PY{n+nn}{np}
\PY{k+kn}{from} \PY{n+nn}{ripser} \PY{k+kn}{import} \PY{n}{ripser}
\PY{k+kn}{from} \PY{n+nn}{persim} \PY{k+kn}{import} \PY{n}{plot\PYZus{}diagrams}
\end{Verbatim}
\end{tcolorbox}

    The functions that we will define in this part will have the following
arguments:

ts\_data = time serie data;

p = the level of the norm \(L^{p}\);

d1 = sliding window width;

d = the dimension of the embedding space;

tho = the speed of the sliding windows: the number of days to skip
between windows;

resolution = resolution in the approximation of persistent landscape;

x\_max = the upper bound of the intervall defining the steps;

x\_min = the lower bound of the intervall defining the steps;

nb\_landscapes = The number of persistent landscapes to consider.

    \hypertarget{data-processing}{%
\subsubsection{Data processing}\label{data-processing}}

\begin{itemize}
\tightlist
\item
  Creation of point cloud data
\end{itemize}

    \begin{tcolorbox}[breakable, size=fbox, boxrule=1pt, pad at break*=1mm,colback=cellbackground, colframe=cellborder]
%\prompt{In}{incolor}{72}{\boxspacing}
\begin{Verbatim}[commandchars=\\\{\}]
\PY{k}{def} \PY{n+nf}{takenEmbedding}\PY{p}{(}\PY{n}{ts\PYZus{}data}\PY{p}{,}\PY{n}{d1}\PY{p}{,}\PY{n}{d}\PY{o}{=}\PY{l+m+mi}{3}\PY{p}{,} \PY{n}{tho}\PY{o}{=}\PY{l+m+mi}{1}\PY{p}{)}\PY{p}{:}
    \PY{n}{array}\PY{o}{=} \PY{n}{np}\PY{o}{.}\PY{n}{array}\PY{p}{(}\PY{n}{ts\PYZus{}data}\PY{p}{)}
    \PY{n}{dic}\PY{o}{=}\PY{p}{\PYZob{}}\PY{p}{\PYZcb{}}
    \PY{k}{if} \PY{n}{d1}\PY{o}{\PYZhy{}}\PY{p}{(}\PY{n}{d}\PY{o}{\PYZhy{}}\PY{l+m+mi}{1}\PY{p}{)}\PY{o}{*}\PY{n}{tho} \PY{o}{\PYZlt{}} \PY{l+m+mi}{0}\PY{p}{:}
        \PY{k}{raise} \PY{n+ne}{ValueError}\PY{p}{(}\PY{l+s+s2}{\PYZdq{}}\PY{l+s+s2}{the time serie len is too small}\PY{l+s+s2}{\PYZdq{}}\PY{p}{)}
    \PY{k}{else}\PY{p}{:}
        \PY{k}{for} \PY{n}{i} \PY{o+ow}{in} \PY{n+nb}{range}\PY{p}{(}\PY{n}{d}\PY{p}{)}\PY{p}{:}
            \PY{n}{dic}\PY{p}{[}\PY{n+nb}{str}\PY{p}{(}\PY{n}{i}\PY{p}{)}\PY{p}{]}\PY{o}{=} \PY{n}{array}\PY{p}{[}\PY{n}{i}\PY{o}{*}\PY{n}{tho}\PY{p}{:} \PY{n}{d1}\PY{o}{\PYZhy{}}\PY{p}{(}\PY{n}{d}\PY{o}{\PYZhy{}}\PY{l+m+mi}{1}\PY{p}{)}\PY{o}{*}\PY{n}{tho}\PY{o}{+}\PY{n}{i}\PY{o}{*}\PY{n}{tho}\PY{p}{]}
    \PY{n}{df}\PY{o}{=}\PY{n}{pd}\PY{o}{.}\PY{n}{DataFrame}\PY{p}{(}\PY{n}{dic}\PY{p}{)}
    \PY{k}{return} \PY{n}{df}
\end{Verbatim}
\end{tcolorbox}

    \hypertarget{vectorization}{%
\subsubsection{Vectorization}\label{vectorization}}

\begin{itemize}
\tightlist
\item
  The persistence landscape function
\end{itemize}

    \begin{tcolorbox}[breakable, size=fbox, boxrule=1pt, pad at break*=1mm,colback=cellbackground, colframe=cellborder]
%\prompt{In}{incolor}{73}{\boxspacing}
\begin{Verbatim}[commandchars=\\\{\}]
\PY{k}{def} \PY{n+nf}{landscapes\PYZus{}approx}\PY{p}{(}\PY{n}{Persistent\PYZus{}diagram}\PY{p}{,}\PY{n}{x\PYZus{}min}\PY{p}{,}\PY{n}{x\PYZus{}max}\PY{p}{,}\PY{n}{resolution}\PY{p}{,}\PY{n}{nb\PYZus{}landscapes}\PY{p}{)}\PY{p}{:}
    \PY{n}{landscape} \PY{o}{=} \PY{n}{np}\PY{o}{.}\PY{n}{zeros}\PY{p}{(}\PY{p}{(}\PY{n}{nb\PYZus{}landscapes}\PY{p}{,}\PY{n}{nb\PYZus{}steps}\PY{p}{)}\PY{p}{)}
    \PY{n}{step} \PY{o}{=} \PY{p}{(}\PY{n}{x\PYZus{}max} \PY{o}{\PYZhy{}} \PY{n}{x\PYZus{}min}\PY{p}{)} \PY{o}{/} \PY{n}{resolution}
    \PY{c+c1}{\PYZsh{}Warning: naive and not the best way to proceed!!!!!}
    \PY{k}{for} \PY{n}{i} \PY{o+ow}{in} \PY{n+nb}{range}\PY{p}{(}\PY{n}{nb\PYZus{}steps}\PY{p}{)}\PY{p}{:}
        \PY{n}{x} \PY{o}{=} \PY{n}{x\PYZus{}min} \PY{o}{+} \PY{n}{i} \PY{o}{*} \PY{n}{step}
        \PY{n}{event\PYZus{}list} \PY{o}{=} \PY{p}{[}\PY{p}{]}
        \PY{k}{for} \PY{n}{pair} \PY{o+ow}{in} \PY{n}{Persistent\PYZus{}diagram}\PY{p}{:}
            \PY{n}{b} \PY{o}{=} \PY{n}{pair}\PY{p}{[}\PY{l+m+mi}{0}\PY{p}{]}
            \PY{n}{d} \PY{o}{=} \PY{n}{pair}\PY{p}{[}\PY{l+m+mi}{1}\PY{p}{]}
            \PY{k}{if} \PY{p}{(}\PY{n}{b} \PY{o}{\PYZlt{}}\PY{o}{=} \PY{n}{x}\PY{p}{)} \PY{o+ow}{and} \PY{p}{(}\PY{n}{x}\PY{o}{\PYZlt{}}\PY{o}{=} \PY{n}{d}\PY{p}{)}\PY{p}{:}
                \PY{k}{if} \PY{n}{x} \PY{o}{\PYZgt{}}\PY{o}{=} \PY{p}{(}\PY{n}{d}\PY{o}{+}\PY{n}{b}\PY{p}{)}\PY{o}{/}\PY{l+m+mf}{2.} \PY{p}{:}
                    \PY{n}{event\PYZus{}list}\PY{o}{.}\PY{n}{append}\PY{p}{(}\PY{p}{(}\PY{n}{d}\PY{o}{\PYZhy{}}\PY{n}{x}\PY{p}{)}\PY{p}{)}
                \PY{k}{else}\PY{p}{:}
                    \PY{n}{event\PYZus{}list}\PY{o}{.}\PY{n}{append}\PY{p}{(}\PY{p}{(}\PY{n}{x}\PY{o}{\PYZhy{}}\PY{n}{b}\PY{p}{)}\PY{p}{)}
        \PY{n}{event\PYZus{}list}\PY{o}{.}\PY{n}{sort}\PY{p}{(}\PY{n}{reverse}\PY{o}{=}\PY{k+kc}{True}\PY{p}{)}
        \PY{n}{event\PYZus{}list} \PY{o}{=} \PY{n}{np}\PY{o}{.}\PY{n}{asarray}\PY{p}{(}\PY{n}{event\PYZus{}list}\PY{p}{)}
        \PY{k}{for} \PY{n}{j} \PY{o+ow}{in} \PY{n+nb}{range}\PY{p}{(}\PY{n}{nb\PYZus{}landscapes}\PY{p}{)}\PY{p}{:}
            \PY{k}{if}\PY{p}{(}\PY{n}{j}\PY{o}{\PYZlt{}}\PY{n+nb}{len}\PY{p}{(}\PY{n}{event\PYZus{}list}\PY{p}{)}\PY{p}{)}\PY{p}{:}
                \PY{n}{landscape}\PY{p}{[}\PY{n}{j}\PY{p}{,}\PY{n}{i}\PY{p}{]}\PY{o}{=}\PY{n}{event\PYZus{}list}\PY{p}{[}\PY{n}{j}\PY{p}{]}
    \PY{k}{return} \PY{n}{landscape}
\end{Verbatim}
\end{tcolorbox}

    \begin{itemize}
\tightlist
\item
  The \(L^p\) norm series function
\end{itemize}

    \begin{tcolorbox}[breakable, size=fbox, boxrule=1pt, pad at break*=1mm,colback=cellbackground, colframe=cellborder]
%\prompt{In}{incolor}{74}{\boxspacing}
\begin{Verbatim}[commandchars=\\\{\}]
\PY{c+c1}{\PYZsh{} The Lp\PYZhy{}norm serie function of a given asset}
\PY{k}{def} \PY{n+nf}{lp\PYZus{}norm\PYZus{}series}\PY{p}{(}\PY{n}{ts\PYZus{}data}\PY{p}{,}\PY{n}{p}\PY{o}{=}\PY{l+m+mi}{2}\PY{p}{,}\PY{n}{d1}\PY{o}{=}\PY{l+m+mi}{20}\PY{p}{,}\PY{n}{d}\PY{o}{=}\PY{l+m+mi}{3}\PY{p}{,} \PY{n}{tho}\PY{o}{=}\PY{l+m+mi}{1}\PY{p}{,} \PY{n}{resolution}\PY{o}{=}\PY{l+m+mi}{1000}\PY{p}{,} \PY{n}{x\PYZus{}max} \PY{o}{=} \PY{l+m+mf}{0.1}\PY{p}{)}\PY{p}{:}
  
    \PY{c+c1}{\PYZsh{} The Lp norm function for a single persistent landscape}
    \PY{k}{def} \PY{n+nf}{lp\PYZus{}norm}\PY{p}{(}\PY{n}{landscape}\PY{p}{,}\PY{n}{p}\PY{p}{)}\PY{p}{:}
        \PY{n}{eta}\PY{o}{=} \PY{n}{landscape}\PY{o}{*}\PY{o}{*}\PY{n}{p}
        \PY{n}{norm}\PY{o}{=} \PY{n}{np}\PY{o}{.}\PY{n}{sum}\PY{p}{(}\PY{n}{eta}\PY{p}{)}
        \PY{k}{return} \PY{n}{norm}
    
    \PY{n}{N}\PY{o}{=} \PY{n+nb}{len}\PY{p}{(}\PY{n}{ts\PYZus{}data}\PY{p}{)}
    \PY{n}{norm\PYZus{}series}\PY{o}{=}\PY{p}{\PYZob{}}\PY{l+s+s1}{\PYZsq{}}\PY{l+s+s1}{H0}\PY{l+s+s1}{\PYZsq{}}\PY{p}{:} \PY{n}{np}\PY{o}{.}\PY{n}{zeros}\PY{p}{(}\PY{n}{N}\PY{o}{\PYZhy{}}\PY{n}{d1}\PY{p}{)}\PY{p}{,} \PY{l+s+s1}{\PYZsq{}}\PY{l+s+s1}{H1}\PY{l+s+s1}{\PYZsq{}}\PY{p}{:} \PY{n}{np}\PY{o}{.}\PY{n}{zeros}\PY{p}{(}\PY{n}{N}\PY{o}{\PYZhy{}}\PY{n}{d1}\PY{p}{)}\PY{p}{\PYZcb{}}
    \PY{k}{for} \PY{n}{k} \PY{o+ow}{in} \PY{n+nb}{range}\PY{p}{(}\PY{n}{N}\PY{p}{)}\PY{p}{:}
        \PY{k}{if} \PY{n}{N}\PY{o}{\PYZhy{}}\PY{n}{k}\PY{o}{\PYZgt{}}\PY{n}{d1}\PY{p}{:}
            \PY{n}{data}\PY{o}{=}\PY{n}{takenEmbedding}\PY{p}{(}\PY{n}{ts\PYZus{}data}\PY{p}{[}\PY{n}{k}\PY{p}{:}\PY{p}{]}\PY{p}{,}\PY{n}{d1}\PY{p}{,}\PY{n}{d}\PY{p}{,} \PY{n}{tho}\PY{o}{=}\PY{l+m+mi}{1}\PY{p}{)}
            \PY{n}{diagrams} \PY{o}{=} \PY{n}{ripser}\PY{p}{(}\PY{n}{data}\PY{p}{)}\PY{p}{[}\PY{l+s+s1}{\PYZsq{}}\PY{l+s+s1}{dgms}\PY{l+s+s1}{\PYZsq{}}\PY{p}{]}
            \PY{n}{landscape0}\PY{o}{=} \PY{n}{landscapes\PYZus{}approx}\PY{p}{(}\PY{n}{diagrams}\PY{p}{[}\PY{l+m+mi}{0}\PY{p}{]}\PY{p}{,}\PY{l+m+mi}{0}\PY{p}{,}\PY{n}{x\PYZus{}max}\PY{p}{,}\PY{n}{resolution}\PY{p}{,}\PY{n+nb}{len}\PY{p}{(}\PY{n}{diagrams}\PY{p}{[}\PY{l+m+mi}{0}\PY{p}{]}\PY{p}{)}\PY{p}{)}
            \PY{n}{landscape1}\PY{o}{=} \PY{n}{landscapes\PYZus{}approx}\PY{p}{(}\PY{n}{diagrams}\PY{p}{[}\PY{l+m+mi}{1}\PY{p}{]}\PY{p}{,}\PY{l+m+mi}{0}\PY{p}{,}\PY{n}{x\PYZus{}max}\PY{p}{,}\PY{n}{resolution}\PY{p}{,}\PY{n+nb}{len}\PY{p}{(}\PY{n}{diagrams}\PY{p}{[}\PY{l+m+mi}{1}\PY{p}{]}\PY{p}{)}\PY{p}{)}
            \PY{n}{norm\PYZus{}series}\PY{p}{[}\PY{l+s+s1}{\PYZsq{}}\PY{l+s+s1}{H0}\PY{l+s+s1}{\PYZsq{}}\PY{p}{]}\PY{p}{[}\PY{n}{k}\PY{p}{]}\PY{o}{=}\PY{n}{lp\PYZus{}norm}\PY{p}{(}\PY{n}{landscape0}\PY{p}{,}\PY{n}{p}\PY{p}{)}
            \PY{n}{norm\PYZus{}series}\PY{p}{[}\PY{l+s+s1}{\PYZsq{}}\PY{l+s+s1}{H1}\PY{l+s+s1}{\PYZsq{}}\PY{p}{]}\PY{p}{[}\PY{n}{k}\PY{p}{]}\PY{o}{=}\PY{n}{lp\PYZus{}norm}\PY{p}{(}\PY{n}{landscape1}\PY{p}{,}\PY{n}{p}\PY{p}{)}
    \PY{k}{return} \PY{n}{norm\PYZus{}series}


\PY{c+c1}{\PYZsh{} The Lp\PYZhy{}norm serie function for the whole portfolio \PYZsq{}Bin\PYZsq{}. Note that dim=\PYZsq{}H0\PYZsq{} or \PYZsq{}H1\PYZsq{}}
\PY{k}{def} \PY{n+nf}{norm\PYZus{}data}\PY{p}{(}\PY{n}{data}\PY{p}{,} \PY{n}{dim}\PY{p}{,} \PY{n}{Bin}\PY{p}{,}\PY{n}{p}\PY{o}{=}\PY{l+m+mi}{2}\PY{p}{,} \PY{n}{d1}\PY{o}{=}\PY{l+m+mi}{20}\PY{p}{,}\PY{n}{d}\PY{o}{=}\PY{l+m+mi}{3}\PY{p}{,} \PY{n}{tho}\PY{o}{=}\PY{l+m+mi}{1}\PY{p}{,} \PY{n}{resolution}\PY{o}{=}\PY{l+m+mi}{1000}\PY{p}{,} \PY{n}{x\PYZus{}max} \PY{o}{=} \PY{l+m+mf}{0.1}\PY{p}{)}\PY{p}{:}
    \PY{n}{norm\PYZus{}data}\PY{o}{=} \PY{p}{\PYZob{}}\PY{p}{\PYZcb{}}
    \PY{k}{for} \PY{n}{i} \PY{o+ow}{in} \PY{n+nb}{range}\PY{p}{(}\PY{n+nb}{len}\PY{p}{(}\PY{n}{Bin}\PY{p}{)}\PY{p}{)}\PY{p}{:}
        \PY{n}{ts\PYZus{}data}\PY{o}{=}\PY{n}{data}\PY{p}{[}\PY{n}{Bin}\PY{p}{[}\PY{n}{i}\PY{p}{]}\PY{p}{]}
        \PY{n}{norm\PYZus{}data}\PY{p}{[}\PY{n}{Bin}\PY{p}{[}\PY{n}{i}\PY{p}{]}\PY{p}{]}\PY{o}{=}\PY{n}{lp\PYZus{}norm\PYZus{}series}\PY{p}{(}\PY{n}{ts\PYZus{}data}\PY{p}{,} \PY{n}{p}\PY{p}{)}\PY{p}{[}\PY{n}{dim}\PY{p}{]}
    \PY{k}{return} \PY{n}{norm\PYZus{}data}
\end{Verbatim}
\end{tcolorbox}

    \hypertarget{portfolio-constructions}{%
\subsubsection{Portfolio constructions}\label{portfolio-constructions}}

We classify assets by their norm values (more precisely, the difference
between their last norm values and their norm mean values). We will next
split the original portfolio into three sub-portfolios: a portfolio
which contains the assets with highest norm values, a portfolio of
assets with lowest norm values and finally and intermediate portfolio
with the rest of assets.

\begin{itemize}
\tightlist
\item
  The TDA enhanced indexing function (TDA\_EI)
\end{itemize}

    \begin{tcolorbox}[breakable, size=fbox, boxrule=1pt, pad at break*=1mm,colback=cellbackground, colframe=cellborder]
%\prompt{In}{incolor}{75}{\boxspacing}
\begin{Verbatim}[commandchars=\\\{\}]
\PY{k}{def} \PY{n+nf}{TDA\PYZus{}EI}\PY{p}{(}\PY{n}{asset\PYZus{}returns}\PY{p}{,} \PY{n}{tickers}\PY{p}{,} \PY{n}{dim}\PY{p}{,}\PY{n}{p}\PY{o}{=}\PY{l+m+mi}{2}\PY{p}{,} \PY{n}{d1}\PY{o}{=}\PY{l+m+mi}{20}\PY{p}{,} \PY{n}{d}\PY{o}{=}\PY{l+m+mi}{3}\PY{p}{,} \PY{n}{tho}\PY{o}{=}\PY{l+m+mi}{1}\PY{p}{)}\PY{p}{:}
    \PY{n}{diff\PYZus{}value}\PY{o}{=} \PY{p}{\PYZob{}}\PY{p}{\PYZcb{}}
    \PY{k}{for} \PY{n}{asset} \PY{o+ow}{in} \PY{n}{tickers}\PY{p}{:}
        \PY{c+c1}{\PYZsh{} Get the data of this asset}
        \PY{n}{ts\PYZus{}data}\PY{o}{=}\PY{n}{asset\PYZus{}returns}\PY{p}{[}\PY{n}{asset}\PY{p}{]}
        \PY{c+c1}{\PYZsh{} Compute the norm series of this asset}
        \PY{n}{norm}\PY{o}{=}\PY{n}{lp\PYZus{}norm\PYZus{}series}\PY{p}{(}\PY{n}{ts\PYZus{}data}\PY{p}{,} \PY{n}{p}\PY{p}{,} \PY{n}{d1}\PY{p}{,} \PY{n}{d}\PY{p}{,} \PY{n}{tho}\PY{p}{)}\PY{p}{[}\PY{n}{dim}\PY{p}{]}
        \PY{c+c1}{\PYZsh{} Find the difference between the last norm value and mean norm value}
        \PY{n}{diff\PYZus{}value}\PY{p}{[}\PY{n}{asset}\PY{p}{]}\PY{o}{=} \PY{n}{norm}\PY{p}{[}\PY{o}{\PYZhy{}}\PY{l+m+mi}{1}\PY{p}{]}\PY{o}{\PYZhy{}}\PY{n}{np}\PY{o}{.}\PY{n}{mean}\PY{p}{(}\PY{n}{norm}\PY{p}{)}
    \PY{c+c1}{\PYZsh{} Sort the diff values of each assets in ascending order}
    \PY{n}{sorted\PYZus{}diff\PYZus{}value}\PY{o}{=} \PY{n+nb}{sorted}\PY{p}{(}\PY{n}{diff\PYZus{}value}\PY{o}{.}\PY{n}{items}\PY{p}{(}\PY{p}{)}\PY{p}{,} \PY{n}{key}\PY{o}{=} \PY{k}{lambda} \PY{n}{kv}\PY{p}{:}\PY{n}{kv}\PY{p}{[}\PY{l+m+mi}{1}\PY{p}{]} \PY{p}{)}
    \PY{n}{N}\PY{o}{=}\PY{n+nb}{len}\PY{p}{(}\PY{n}{sorted\PYZus{}diff\PYZus{}value}\PY{p}{)}
    \PY{c+c1}{\PYZsh{} Devide the asset class in three categories: bin1, bin2 and \PYZsq{}bin3\PYZsq{}}
    \PY{n}{bin1}\PY{o}{=}\PY{p}{[}\PY{n}{v}\PY{p}{[}\PY{l+m+mi}{0}\PY{p}{]} \PY{k}{for} \PY{n}{v} \PY{o+ow}{in} \PY{n}{sorted\PYZus{}diff\PYZus{}value}\PY{p}{[}\PY{p}{:}\PY{n+nb}{round}\PY{p}{(}\PY{n}{N}\PY{o}{/}\PY{l+m+mi}{3}\PY{p}{)}\PY{p}{]}\PY{p}{]}
    \PY{n}{bin2}\PY{o}{=}\PY{p}{[}\PY{n}{v}\PY{p}{[}\PY{l+m+mi}{0}\PY{p}{]} \PY{k}{for} \PY{n}{v} \PY{o+ow}{in} \PY{n}{sorted\PYZus{}diff\PYZus{}value}\PY{p}{[}\PY{n+nb}{round}\PY{p}{(}\PY{n}{N}\PY{o}{/}\PY{l+m+mi}{3}\PY{p}{)}\PY{p}{:}\PY{n+nb}{round}\PY{p}{(}\PY{l+m+mi}{2}\PY{o}{*}\PY{n}{N}\PY{o}{/}\PY{l+m+mi}{3}\PY{p}{)}\PY{p}{]}\PY{p}{]}
    \PY{n}{bin3}\PY{o}{=}\PY{p}{[}\PY{n}{v}\PY{p}{[}\PY{l+m+mi}{0}\PY{p}{]} \PY{k}{for} \PY{n}{v} \PY{o+ow}{in} \PY{n}{sorted\PYZus{}diff\PYZus{}value}\PY{p}{[}\PY{n+nb}{round}\PY{p}{(}\PY{l+m+mi}{2}\PY{o}{*}\PY{n}{N}\PY{o}{/}\PY{l+m+mi}{3}\PY{p}{)}\PY{p}{:} \PY{n}{N}\PY{p}{]}\PY{p}{]}
    \PY{k}{return} \PY{p}{\PYZob{}}\PY{l+s+s1}{\PYZsq{}}\PY{l+s+s1}{bin1}\PY{l+s+s1}{\PYZsq{}}\PY{p}{:} \PY{n}{bin1}\PY{p}{,} \PY{l+s+s1}{\PYZsq{}}\PY{l+s+s1}{bin2}\PY{l+s+s1}{\PYZsq{}}\PY{p}{:} \PY{n}{bin2}\PY{p}{,} \PY{l+s+s1}{\PYZsq{}}\PY{l+s+s1}{bin3}\PY{l+s+s1}{\PYZsq{}}\PY{p}{:} \PY{n}{bin3}\PY{p}{\PYZcb{}}
\end{Verbatim}
\end{tcolorbox}

    \begin{itemize}
\tightlist
\item
  Splitting the original portfolio into sub-portfolios
\end{itemize}

    \begin{tcolorbox}[breakable, size=fbox, boxrule=1pt, pad at break*=1mm,colback=cellbackground, colframe=cellborder]
%\prompt{In}{incolor}{76}{\boxspacing}
\begin{Verbatim}[commandchars=\\\{\}]
\PY{n}{bins}\PY{o}{=} \PY{n}{TDA\PYZus{}EI}\PY{p}{(}\PY{n}{asset\PYZus{}returns}\PY{p}{,} \PY{n}{tickers}\PY{p}{,} \PY{l+s+s1}{\PYZsq{}}\PY{l+s+s1}{H1}\PY{l+s+s1}{\PYZsq{}}\PY{p}{,} \PY{l+m+mi}{3}\PY{p}{,} \PY{n}{d1}\PY{o}{=}\PY{l+m+mi}{20}\PY{p}{,} \PY{n}{d}\PY{o}{=}\PY{l+m+mi}{3}\PY{p}{,} \PY{n}{tho}\PY{o}{=}\PY{l+m+mi}{1}\PY{p}{)}
\PY{n+nb}{print}\PY{p}{(}\PY{n}{bins}\PY{p}{)}
\end{Verbatim}
\end{tcolorbox}

    \begin{Verbatim}[commandchars=\\\{\}]
\{'bin1': ['GLE.PA', 'ML.PA', 'ENGI.PA', 'TEP.PA', 'EN.PA', 'BNP.PA', 'VIE.PA',
'CA.PA', 'KER.PA', 'SU.PA'], 'bin2': ['MC.PA', 'VIV.PA', 'FP.PA', 'DG.PA',
'LR.PA', 'BN.PA', 'OR.PA', 'CAP.PA', 'AI.PA', 'SAN.PA'], 'bin3': ['RI.PA',
'ATO.PA', 'ORA.PA', 'AIR.PA', 'AC.PA', 'UG.PA', 'SW.PA', 'SGO.PA', 'ACA.PA',
'HO.PA']\}
    \end{Verbatim}

    \hypertarget{financial-analysis-of-tda-portfolios}{%
\section{Financial analysis of TDA
portfolios}\label{financial-analysis-of-tda-portfolios}}

In the rest of this paper, we will compare the risk and portfolio's
return of bin1 and bin3 and ignore bin2 which can be seen as an
intermediate portfolio.

We will see that

\begin{itemize}
\tightlist
\item
  bin1 is suitable for people with high-risk profile
\item
  bin3 is suitable for people with low-risk profile
\end{itemize}

\hypertarget{risk-analysis}{%
\subsection{Risk analysis}\label{risk-analysis}}

\hypertarget{computing-the-risk-cvar-of-stocks-per-portfolio}{%
\subsubsection{Computing the risk (CVaR) of stocks per
portfolio}\label{computing-the-risk-cvar-of-stocks-per-portfolio}}

    \begin{tcolorbox}[breakable, size=fbox, boxrule=1pt, pad at break*=1mm,colback=cellbackground, colframe=cellborder]
%\prompt{In}{incolor}{77}{\boxspacing}
\begin{Verbatim}[commandchars=\\\{\}]
\PY{k+kn}{from} \PY{n+nn}{scipy}\PY{n+nn}{.}\PY{n+nn}{stats} \PY{k+kn}{import} \PY{n}{norm}

\PY{c+c1}{\PYZsh{} The conditional value at risk (CVaR) function of a single asset}
\PY{k}{def} \PY{n+nf}{CVar}\PY{p}{(}\PY{n}{asset\PYZus{}return}\PY{p}{,} \PY{n}{conf}\PY{o}{=}\PY{l+m+mf}{0.95}\PY{p}{)}\PY{p}{:}
    \PY{n}{loss}\PY{o}{=} \PY{o}{\PYZhy{}}\PY{n}{asset\PYZus{}return}
    \PY{c+c1}{\PYZsh{} Compute the mean and the variance of the portfolio}
    \PY{n}{pm}\PY{o}{=}\PY{n}{loss}\PY{o}{.}\PY{n}{mean}\PY{p}{(}\PY{p}{)}
    \PY{n}{ps}\PY{o}{=}\PY{n}{loss}\PY{o}{.}\PY{n}{std}\PY{p}{(}\PY{p}{)}
    \PY{c+c1}{\PYZsh{} Compute the conf\PYZpc{} VaR using .ppf()}
    \PY{n}{VaR\PYZus{}conf} \PY{o}{=} \PY{n}{norm}\PY{o}{.}\PY{n}{ppf}\PY{p}{(}\PY{n}{conf}\PY{p}{,} \PY{n}{loc} \PY{o}{=} \PY{n}{pm}\PY{p}{,} \PY{n}{scale} \PY{o}{=} \PY{n}{ps}\PY{p}{)}
    \PY{c+c1}{\PYZsh{} Compute the expected tail loss and the CVaR in the worst 5\PYZpc{} of cases}
    \PY{n}{tail\PYZus{}loss} \PY{o}{=} \PY{n}{norm}\PY{o}{.}\PY{n}{expect}\PY{p}{(}\PY{k}{lambda} \PY{n}{x}\PY{p}{:} \PY{n}{x}\PY{p}{,} \PY{n}{loc} \PY{o}{=} \PY{n}{pm}\PY{p}{,} \PY{n}{scale} \PY{o}{=} \PY{n}{ps}\PY{p}{,} \PY{n}{lb} \PY{o}{=} \PY{n}{VaR\PYZus{}conf} \PY{p}{)}
    \PY{n}{CVar\PYZus{}conf} \PY{o}{=} \PY{p}{(}\PY{l+m+mi}{1} \PY{o}{/} \PY{p}{(}\PY{l+m+mi}{1} \PY{o}{\PYZhy{}} \PY{n}{conf}\PY{p}{)}\PY{p}{)} \PY{o}{*} \PY{n}{tail\PYZus{}loss}
    \PY{k}{return} \PY{n}{CVar\PYZus{}conf}

\PY{c+c1}{\PYZsh{} The conditional value at risk (CVaR) function of a portfolio}
\PY{k}{def} \PY{n+nf}{CVar\PYZus{}portfolio}\PY{p}{(}\PY{n}{tickers}\PY{p}{,} \PY{n}{asset\PYZus{}returns}\PY{p}{,} \PY{n}{conf}\PY{o}{=}\PY{l+m+mf}{0.95}\PY{p}{)}\PY{p}{:}
    \PY{n}{CVar\PYZus{}portfolio}\PY{o}{=}\PY{p}{\PYZob{}}\PY{p}{\PYZcb{}}
    \PY{k}{for} \PY{n}{asset} \PY{o+ow}{in} \PY{n}{tickers}\PY{p}{:}
        \PY{n}{asset\PYZus{}return}\PY{o}{=} \PY{n}{asset\PYZus{}returns}\PY{p}{[}\PY{n}{asset}\PY{p}{]}
        \PY{n}{CVar\PYZus{}portfolio}\PY{p}{[}\PY{n}{asset}\PY{p}{]}\PY{o}{=}\PY{n}{CVar}\PY{p}{(}\PY{n}{asset\PYZus{}return}\PY{p}{,} \PY{n}{conf}\PY{p}{)}
    \PY{k}{return} \PY{n}{CVar\PYZus{}portfolio}

\PY{c+c1}{\PYZsh{} Computing the conditional values at risk (CVaR) of the different assets in each portfolio      }
\PY{n}{CVar\PYZus{}Bin1}\PY{o}{=} \PY{n}{CVar\PYZus{}portfolio}\PY{p}{(}\PY{n}{bins}\PY{p}{[}\PY{l+s+s1}{\PYZsq{}}\PY{l+s+s1}{bin1}\PY{l+s+s1}{\PYZsq{}}\PY{p}{]}\PY{p}{,} \PY{n}{asset\PYZus{}returns}\PY{p}{)} 
\PY{n}{CVar\PYZus{}Bin2}\PY{o}{=} \PY{n}{CVar\PYZus{}portfolio}\PY{p}{(}\PY{n}{bins}\PY{p}{[}\PY{l+s+s1}{\PYZsq{}}\PY{l+s+s1}{bin2}\PY{l+s+s1}{\PYZsq{}}\PY{p}{]}\PY{p}{,} \PY{n}{asset\PYZus{}returns}\PY{p}{)} 
\PY{n}{CVar\PYZus{}Bin3}\PY{o}{=} \PY{n}{CVar\PYZus{}portfolio}\PY{p}{(}\PY{n}{bins}\PY{p}{[}\PY{l+s+s1}{\PYZsq{}}\PY{l+s+s1}{bin3}\PY{l+s+s1}{\PYZsq{}}\PY{p}{]}\PY{p}{,} \PY{n}{asset\PYZus{}returns}\PY{p}{)} 
\end{Verbatim}
\end{tcolorbox}

    \hypertarget{representation-of-cvar-of-assets-per-sub-portfolios}{%
\subsubsection{Representation of CVaR of assets per
sub-portfolios}\label{representation-of-cvar-of-assets-per-sub-portfolios}}

    \begin{tcolorbox}[breakable, size=fbox, boxrule=1pt, pad at break*=1mm,colback=cellbackground, colframe=cellborder]
%\prompt{In}{incolor}{90}{\boxspacing}
\begin{Verbatim}[commandchars=\\\{\}]
\PY{n}{lists1} \PY{o}{=} \PY{n+nb}{sorted}\PY{p}{(}\PY{n}{CVar\PYZus{}Bin1}\PY{o}{.}\PY{n}{items}\PY{p}{(}\PY{p}{)}\PY{p}{)} \PY{c+c1}{\PYZsh{} sorted by key, return a list of tuples}
\PY{n}{x1}\PY{p}{,} \PY{n}{y1} \PY{o}{=} \PY{n+nb}{zip}\PY{p}{(}\PY{o}{*}\PY{n}{lists1}\PY{p}{)}


\PY{n}{lists2} \PY{o}{=} \PY{n+nb}{sorted}\PY{p}{(}\PY{n}{CVar\PYZus{}Bin2}\PY{o}{.}\PY{n}{items}\PY{p}{(}\PY{p}{)}\PY{p}{)} \PY{c+c1}{\PYZsh{} sorted by key, return a list of tuples}
\PY{n}{x2}\PY{p}{,} \PY{n}{y2} \PY{o}{=} \PY{n+nb}{zip}\PY{p}{(}\PY{o}{*}\PY{n}{lists2}\PY{p}{)}

\PY{n}{lists3} \PY{o}{=} \PY{n+nb}{sorted}\PY{p}{(}\PY{n}{CVar\PYZus{}Bin3}\PY{o}{.}\PY{n}{items}\PY{p}{(}\PY{p}{)}\PY{p}{)} \PY{c+c1}{\PYZsh{} sorted by key, return a list of tuples}
\PY{n}{x3}\PY{p}{,} \PY{n}{y3} \PY{o}{=} \PY{n+nb}{zip}\PY{p}{(}\PY{o}{*}\PY{n}{lists3}\PY{p}{)}

\PY{n}{plt}\PY{o}{.}\PY{n}{style}\PY{o}{.}\PY{n}{use}\PY{p}{(}\PY{l+s+s1}{\PYZsq{}}\PY{l+s+s1}{seaborn}\PY{l+s+s1}{\PYZsq{}}\PY{p}{)}
\PY{n}{plt}\PY{o}{.}\PY{n}{plot}\PY{p}{(}\PY{n}{x1}\PY{p}{,} \PY{n}{y1}\PY{p}{,} \PY{l+s+s1}{\PYZsq{}}\PY{l+s+s1}{o}\PY{l+s+s1}{\PYZsq{}}\PY{p}{,} \PY{n}{label}\PY{o}{=}\PY{l+s+s1}{\PYZsq{}}\PY{l+s+s1}{bin1}\PY{l+s+s1}{\PYZsq{}}\PY{p}{)}
\PY{n}{plt}\PY{o}{.}\PY{n}{plot}\PY{p}{(}\PY{n}{x2}\PY{p}{,} \PY{n}{y2}\PY{p}{,} \PY{l+s+s1}{\PYZsq{}}\PY{l+s+s1}{o}\PY{l+s+s1}{\PYZsq{}}\PY{p}{,} \PY{n}{label}\PY{o}{=}\PY{l+s+s1}{\PYZsq{}}\PY{l+s+s1}{bin2}\PY{l+s+s1}{\PYZsq{}}\PY{p}{)}
\PY{n}{plt}\PY{o}{.}\PY{n}{plot}\PY{p}{(}\PY{n}{x3}\PY{p}{,} \PY{n}{y3}\PY{p}{,} \PY{l+s+s1}{\PYZsq{}}\PY{l+s+s1}{o}\PY{l+s+s1}{\PYZsq{}}\PY{p}{,} \PY{n}{label}\PY{o}{=}\PY{l+s+s1}{\PYZsq{}}\PY{l+s+s1}{bin3}\PY{l+s+s1}{\PYZsq{}}\PY{p}{)}
\PY{n}{plt}\PY{o}{.}\PY{n}{legend}\PY{p}{(}\PY{n}{loc}\PY{o}{=}\PY{l+s+s1}{\PYZsq{}}\PY{l+s+s1}{best}\PY{l+s+s1}{\PYZsq{}}\PY{p}{)}
\PY{n}{plt}\PY{o}{.}\PY{n}{ylabel}\PY{p}{(}\PY{l+s+s1}{\PYZsq{}}\PY{l+s+s1}{CVaR}\PY{l+s+s1}{\PYZsq{}}\PY{p}{,} \PY{n}{fontsize}\PY{o}{=}\PY{l+m+mi}{20}\PY{p}{)} 
\PY{n}{plt}\PY{o}{.}\PY{n}{xticks}\PY{p}{(}\PY{p}{[}\PY{p}{]}\PY{p}{)}
\PY{n}{plt}\PY{o}{.}\PY{n}{show}\PY{p}{(}\PY{p}{)}
\end{Verbatim}
\end{tcolorbox}

    \begin{center}
    \adjustimage{max size={0.9\linewidth}{0.9\paperheight}}{output_29_0.png}
    \end{center}
    { \hspace*{\fill} \\}
    
    \hypertarget{computing-the-risk-mean-per-sub-portfolios}{%
\subsubsection{Computing the risk mean per
sub-portfolios}\label{computing-the-risk-mean-per-sub-portfolios}}

    \begin{tcolorbox}[breakable, size=fbox, boxrule=1pt, pad at break*=1mm,colback=cellbackground, colframe=cellborder]
%\prompt{In}{incolor}{92}{\boxspacing}
\begin{Verbatim}[commandchars=\\\{\}]
\PY{k+kn}{from} \PY{n+nn}{numpy} \PY{k+kn}{import} \PY{n}{array}
\PY{n}{CVar\PYZus{}risk1}\PY{o}{=}\PY{n}{array}\PY{p}{(}\PY{p}{[}\PY{n}{CVar\PYZus{}Bin1}\PY{p}{[}\PY{n}{k}\PY{p}{]} \PY{k}{for} \PY{n}{k} \PY{o+ow}{in} \PY{n}{CVar\PYZus{}Bin1}\PY{p}{]}\PY{p}{)}\PY{o}{.}\PY{n}{mean}\PY{p}{(}\PY{p}{)}
\PY{n}{CVar\PYZus{}risk2}\PY{o}{=}\PY{n}{array}\PY{p}{(}\PY{p}{[}\PY{n}{CVar\PYZus{}Bin2}\PY{p}{[}\PY{n}{k}\PY{p}{]} \PY{k}{for} \PY{n}{k} \PY{o+ow}{in} \PY{n}{CVar\PYZus{}Bin2}\PY{p}{]}\PY{p}{)}\PY{o}{.}\PY{n}{mean}\PY{p}{(}\PY{p}{)}
\PY{n}{CVar\PYZus{}risk3}\PY{o}{=}\PY{n}{array}\PY{p}{(}\PY{p}{[}\PY{n}{CVar\PYZus{}Bin3}\PY{p}{[}\PY{n}{k}\PY{p}{]} \PY{k}{for} \PY{n}{k} \PY{o+ow}{in} \PY{n}{CVar\PYZus{}Bin3}\PY{p}{]}\PY{p}{)}\PY{o}{.}\PY{n}{mean}\PY{p}{(}\PY{p}{)}
\PY{n+nb}{print}\PY{p}{(}\PY{l+s+s1}{\PYZsq{}}\PY{l+s+s1}{CVaR mean of bin1:}\PY{l+s+s1}{\PYZsq{}}\PY{p}{,}\PY{n}{CVar\PYZus{}risk1}\PY{p}{)}
\PY{n+nb}{print}\PY{p}{(}\PY{l+s+s1}{\PYZsq{}}\PY{l+s+s1}{CVaR mean of bin2:}\PY{l+s+s1}{\PYZsq{}}\PY{p}{,}\PY{n}{CVar\PYZus{}risk2}\PY{p}{)}
\PY{n+nb}{print}\PY{p}{(}\PY{l+s+s1}{\PYZsq{}}\PY{l+s+s1}{CVaR mean of bin3:}\PY{l+s+s1}{\PYZsq{}}\PY{p}{,}\PY{n}{CVar\PYZus{}risk3}\PY{p}{)}
\end{Verbatim}
\end{tcolorbox}

    \begin{Verbatim}[commandchars=\\\{\}]
CVaR mean of bin1: 0.037135837048476866
CVaR mean of bin2: 0.030226966973985865
CVaR mean of bin3: 0.03723000647791284
    \end{Verbatim}

    We deduce from this risk analysis that even if bin1 tends to have lower
risk mean than bin3, the plot shows that the CVaR in bin1 is relatively
high (for all its assets) while the dispersion of risk values in bin3 is
higher. In other words, it is less risky to invest on bin3 than
investing on bin1.

    \hypertarget{portfolio-optimization}{%
\subsection{Portfolio optimization}\label{portfolio-optimization}}

In this part, we compute the portfolio weights in bin1 and bin3. Then we
compute the expected return of these sub-portfolios.

\hypertarget{building-portfolio-weight}{%
\subsubsection{Building portfolio
weight}\label{building-portfolio-weight}}

The goal is to solve the optimization problem ( see {[}3{]} for a
complete description )

\[ \underset{w}{min} \underset{t=1}{\overset{d_1 -20}{\Sigma}}  \underset{t=1}{\overset{n}{\Sigma}} |N_{it}w_i-N_t|\]
subject to
\[\underset{i=1}{\overset{n}{\Sigma}}\mu_i w_i -\mu_I \geq r^*\],
\[ \underset{i=1}{\overset{n}{\Sigma}}w_i =1,\]

\[w_i \geq 0, i=1,..., n\]

where - \(N_t:\) the TDA-norm of the benchmark index - \(N_{it}:\) the
TDA-norm of the benchmark asset \(i, i=1,..., n\) - \(\mu_I:\) the
expected return from the benchmark index -
\(\mu_i= \frac{1}{d_1} \Sigma_{t=1}^{d_1} x_{it}:\) the expected return
from the benchmark asset \(i\) in the \(d_1\) days of the sample period
- \(r^{*}\) is the desired return over the above benchmark return from
the portfolio

    \begin{tcolorbox}[breakable, size=fbox, boxrule=1pt, pad at break*=1mm,colback=cellbackground, colframe=cellborder]
%\prompt{In}{incolor}{80}{\boxspacing}
\begin{Verbatim}[commandchars=\\\{\}]
\PY{c+c1}{\PYZsh{} Optimisation: optimal weight function}
\PY{k}{def} \PY{n+nf}{optimal\PYZus{}weight}\PY{p}{(}\PY{n}{Bin}\PY{p}{,} \PY{n}{asset\PYZus{}returns}\PY{p}{,}\PY{n}{index}\PY{p}{,} \PY{n}{dim}\PY{p}{,} \PY{n}{p}\PY{o}{=}\PY{l+m+mi}{2}\PY{p}{,}\PY{n}{r}\PY{o}{=}\PY{l+m+mf}{0.02}\PY{p}{)}\PY{p}{:}
    \PY{n}{Bin}\PY{o}{=}\PY{n}{Bin}
    \PY{n}{mu\PYZus{}assets}\PY{o}{=} \PY{n}{np}\PY{o}{.}\PY{n}{array}\PY{p}{(}\PY{n}{asset\PYZus{}returns}\PY{p}{[}\PY{n}{Bin}\PY{p}{]}\PY{o}{.}\PY{n}{mean}\PY{p}{(}\PY{p}{)}\PY{p}{)}
    \PY{n}{mu\PYZus{}index} \PY{o}{=} \PY{n}{np}\PY{o}{.}\PY{n}{array}\PY{p}{(}\PY{n}{asset\PYZus{}returns}\PY{p}{[}\PY{n}{index}\PY{p}{]}\PY{o}{.}\PY{n}{mean}\PY{p}{(}\PY{p}{)}\PY{p}{)}
    \PY{n}{Norm\PYZus{}data}\PY{o}{=} \PY{n}{norm\PYZus{}data}\PY{p}{(}\PY{n}{asset\PYZus{}returns}\PY{p}{[}\PY{n}{Bin}\PY{p}{]}\PY{p}{,} \PY{n}{dim}\PY{p}{,} \PY{n}{Bin}\PY{p}{)}
    \PY{n}{N}\PY{o}{=} \PY{n}{lp\PYZus{}norm\PYZus{}series}\PY{p}{(}\PY{n}{asset\PYZus{}returns}\PY{p}{[}\PY{n}{index}\PY{p}{]}\PY{p}{,} \PY{n}{p}\PY{p}{)}\PY{p}{[}\PY{n}{dim}\PY{p}{]} 
    
    \PY{c+c1}{\PYZsh{} Definition of constraints}
    \PY{k}{def} \PY{n+nf}{constraint1}\PY{p}{(}\PY{n}{w}\PY{p}{)}\PY{p}{:}
        \PY{k}{return} \PY{n}{mu\PYZus{}assets}\PY{o}{.}\PY{n}{dot}\PY{p}{(}\PY{n}{w}\PY{p}{)}\PY{o}{\PYZhy{}}\PY{n}{mu\PYZus{}index}\PY{o}{\PYZhy{}}\PY{n}{r}
    \PY{k}{def} \PY{n+nf}{constraint2}\PY{p}{(}\PY{n}{w}\PY{p}{)}\PY{p}{:}
        \PY{k}{return} \PY{n}{np}\PY{o}{.}\PY{n}{sum}\PY{p}{(}\PY{n}{w}\PY{p}{)}\PY{o}{\PYZhy{}}\PY{l+m+mi}{1}
    \PY{n}{con1}\PY{o}{=} \PY{p}{\PYZob{}}\PY{l+s+s1}{\PYZsq{}}\PY{l+s+s1}{type}\PY{l+s+s1}{\PYZsq{}}\PY{p}{:} \PY{l+s+s1}{\PYZsq{}}\PY{l+s+s1}{ineq}\PY{l+s+s1}{\PYZsq{}}\PY{p}{,} \PY{l+s+s1}{\PYZsq{}}\PY{l+s+s1}{fun}\PY{l+s+s1}{\PYZsq{}}\PY{p}{:}\PY{n}{constraint1}\PY{p}{\PYZcb{}}
    \PY{n}{con2}\PY{o}{=} \PY{p}{\PYZob{}}\PY{l+s+s1}{\PYZsq{}}\PY{l+s+s1}{type}\PY{l+s+s1}{\PYZsq{}}\PY{p}{:} \PY{l+s+s1}{\PYZsq{}}\PY{l+s+s1}{eq}\PY{l+s+s1}{\PYZsq{}}\PY{p}{,} \PY{l+s+s1}{\PYZsq{}}\PY{l+s+s1}{fun}\PY{l+s+s1}{\PYZsq{}}\PY{p}{:}\PY{n}{constraint2}\PY{p}{\PYZcb{}}
    \PY{n}{cons}\PY{o}{=} \PY{p}{[}\PY{n}{con1}\PY{p}{,} \PY{n}{con2}\PY{p}{]}
    
    \PY{c+c1}{\PYZsh{} definition of bounds}
    \PY{n}{zeros}\PY{o}{=} \PY{n}{np}\PY{o}{.}\PY{n}{zeros}\PY{p}{(}\PY{n+nb}{len}\PY{p}{(}\PY{n}{Bin}\PY{p}{)}\PY{p}{)}
    \PY{n}{ones}\PY{o}{=} \PY{n}{np}\PY{o}{.}\PY{n}{ones}\PY{p}{(}\PY{n+nb}{len}\PY{p}{(}\PY{n}{Bin}\PY{p}{)}\PY{p}{)}
    \PY{n}{bounds} \PY{o}{=} \PY{n}{Bounds}\PY{p}{(}\PY{n}{zeros}\PY{o}{.}\PY{n}{tolist}\PY{p}{(}\PY{p}{)}\PY{p}{,} \PY{n}{ones}\PY{o}{.}\PY{n}{tolist}\PY{p}{(}\PY{p}{)}\PY{p}{)}
    
    \PY{c+c1}{\PYZsh{} Objective function for optimisation}
    \PY{k}{def} \PY{n+nf}{ETDA\PYZus{}func}\PY{p}{(}\PY{n}{w}\PY{p}{)}\PY{p}{:}
        \PY{n}{part}\PY{o}{=}\PY{l+m+mi}{0}
        \PY{k}{for} \PY{n}{i} \PY{o+ow}{in} \PY{n+nb}{range}\PY{p}{(}\PY{n+nb}{len}\PY{p}{(}\PY{n}{Bin}\PY{p}{)}\PY{p}{)}\PY{p}{:}
            \PY{n}{norm}\PY{o}{=}\PY{n}{Norm\PYZus{}data}\PY{p}{[}\PY{n}{Bin}\PY{p}{[}\PY{n}{i}\PY{p}{]}\PY{p}{]}
            \PY{n}{part}\PY{o}{+}\PY{o}{=} \PY{n}{np}\PY{o}{.}\PY{n}{sum}\PY{p}{(}\PY{n}{np}\PY{o}{.}\PY{n}{abs}\PY{p}{(}\PY{n}{norm}\PY{o}{*}\PY{n}{w}\PY{p}{[}\PY{n}{i}\PY{p}{]}\PY{o}{\PYZhy{}}\PY{n}{N}\PY{p}{)}\PY{p}{)}
        \PY{k}{return} \PY{n}{part}
    
    \PY{c+c1}{\PYZsh{} Optimisation}
    \PY{n}{w0}\PY{o}{=}\PY{n}{np}\PY{o}{.}\PY{n}{ones}\PY{p}{(}\PY{n+nb}{len}\PY{p}{(}\PY{n}{Bin}\PY{p}{)}\PY{p}{)}
    \PY{n}{sol}\PY{o}{=} \PY{n}{minimize}\PY{p}{(}\PY{n}{ETDA\PYZus{}func}\PY{p}{,} \PY{n}{w0}\PY{p}{,} \PY{n}{method}\PY{o}{=}\PY{l+s+s1}{\PYZsq{}}\PY{l+s+s1}{SLSQP}\PY{l+s+s1}{\PYZsq{}}\PY{p}{,} \PY{n}{bounds}\PY{o}{=}\PY{n}{bounds}\PY{p}{,} \PY{n}{constraints}\PY{o}{=} \PY{n}{cons}\PY{p}{)}
    
    \PY{k}{return} \PY{n}{sol}\PY{o}{.}\PY{n}{x}
\end{Verbatim}
\end{tcolorbox}

    \begin{tcolorbox}[breakable, size=fbox, boxrule=1pt, pad at break*=1mm,colback=cellbackground, colframe=cellborder]
%\prompt{In}{incolor}{85}{\boxspacing}
\begin{Verbatim}[commandchars=\\\{\}]
\PY{c+c1}{\PYZsh{} We assume in this paper that r=0}
\PY{n}{weight\PYZus{}bin1}\PY{o}{=}\PY{n}{optimal\PYZus{}weight}\PY{p}{(}\PY{n}{bins}\PY{p}{[}\PY{l+s+s1}{\PYZsq{}}\PY{l+s+s1}{bin1}\PY{l+s+s1}{\PYZsq{}}\PY{p}{]}\PY{p}{,} \PY{n}{asset\PYZus{}returns}\PY{p}{,}\PY{l+s+s1}{\PYZsq{}}\PY{l+s+s1}{\PYZca{}FCHI}\PY{l+s+s1}{\PYZsq{}}\PY{p}{,} \PY{l+s+s1}{\PYZsq{}}\PY{l+s+s1}{H1}\PY{l+s+s1}{\PYZsq{}}\PY{p}{,}\PY{l+m+mi}{3}\PY{p}{,}\PY{l+m+mi}{0}\PY{p}{)}
\PY{n}{weight\PYZus{}bin3}\PY{o}{=}\PY{n}{optimal\PYZus{}weight}\PY{p}{(}\PY{n}{bins}\PY{p}{[}\PY{l+s+s1}{\PYZsq{}}\PY{l+s+s1}{bin3}\PY{l+s+s1}{\PYZsq{}}\PY{p}{]}\PY{p}{,} \PY{n}{asset\PYZus{}returns}\PY{p}{,}\PY{l+s+s1}{\PYZsq{}}\PY{l+s+s1}{\PYZca{}FCHI}\PY{l+s+s1}{\PYZsq{}}\PY{p}{,} \PY{l+s+s1}{\PYZsq{}}\PY{l+s+s1}{H1}\PY{l+s+s1}{\PYZsq{}}\PY{p}{,}\PY{l+m+mi}{3}\PY{p}{,}\PY{l+m+mi}{0}\PY{p}{)}
\end{Verbatim}
\end{tcolorbox}

    \begin{tcolorbox}[breakable, size=fbox, boxrule=1pt, pad at break*=1mm,colback=cellbackground, colframe=cellborder]
%\prompt{In}{incolor}{91}{\boxspacing}
\begin{Verbatim}[commandchars=\\\{\}]
\PY{n+nb}{print}\PY{p}{(}\PY{l+s+s1}{\PYZsq{}}\PY{l+s+s1}{weight for bin1:}\PY{l+s+s1}{\PYZsq{}}\PY{p}{,} \PY{n}{weight\PYZus{}bin1}\PY{p}{)}
\PY{n+nb}{print}\PY{p}{(}\PY{l+s+s1}{\PYZsq{}}\PY{l+s+s1}{weight for bin3:}\PY{l+s+s1}{\PYZsq{}}\PY{p}{,} \PY{n}{weight\PYZus{}bin3}\PY{p}{)}
\end{Verbatim}
\end{tcolorbox}

    \begin{Verbatim}[commandchars=\\\{\}]
weight for bin1: [4.04002103e-05 3.64094537e-04 2.65729927e-03 8.56415312e-04
 1.12546878e-03 1.06023060e-04 2.99057950e-03 2.12413177e-03
 9.54150575e-01 3.55850126e-02]
weight for bin3: [9.90907663e-01 6.24142404e-04 1.60937350e-03 1.72645434e-04
 8.20194429e-04 1.24489715e-05 3.87459882e-03 7.90584201e-04
 3.65140800e-05 1.15183497e-03]
    \end{Verbatim}

    \hypertarget{building-portfolio-returns}{%
\subsubsection{Building portfolio
returns}\label{building-portfolio-returns}}

    \begin{tcolorbox}[breakable, size=fbox, boxrule=1pt, pad at break*=1mm,colback=cellbackground, colframe=cellborder]
%\prompt{In}{incolor}{88}{\boxspacing}
\begin{Verbatim}[commandchars=\\\{\}]
\PY{c+c1}{\PYZsh{} Portfolio return}
\PY{n}{return\PYZus{}bin1}\PY{o}{=}\PY{n}{asset\PYZus{}returns}\PY{p}{[}\PY{n}{bins}\PY{p}{[}\PY{l+s+s1}{\PYZsq{}}\PY{l+s+s1}{bin1}\PY{l+s+s1}{\PYZsq{}}\PY{p}{]}\PY{p}{]}\PY{o}{.}\PY{n}{dot}\PY{p}{(}\PY{n}{weight\PYZus{}bin1}\PY{p}{)}
\PY{n}{return\PYZus{}bin3}\PY{o}{=}\PY{n}{asset\PYZus{}returns}\PY{p}{[}\PY{n}{bins}\PY{p}{[}\PY{l+s+s1}{\PYZsq{}}\PY{l+s+s1}{bin3}\PY{l+s+s1}{\PYZsq{}}\PY{p}{]}\PY{p}{]}\PY{o}{.}\PY{n}{dot}\PY{p}{(}\PY{n}{weight\PYZus{}bin3}\PY{p}{)}

\PY{c+c1}{\PYZsh{} Cumulative portfolio returns}
\PY{n}{cumulative\PYZus{}ret\PYZus{}bin1} \PY{o}{=} \PY{p}{(}\PY{p}{(}\PY{n}{return\PYZus{}bin1}\PY{o}{+}\PY{l+m+mi}{1}\PY{p}{)}\PY{o}{.}\PY{n}{cumprod}\PY{p}{(}\PY{p}{)}\PY{o}{\PYZhy{}}\PY{l+m+mi}{1}\PY{p}{)}
\PY{n}{cumulative\PYZus{}ret\PYZus{}bin3} \PY{o}{=} \PY{p}{(}\PY{p}{(}\PY{n}{return\PYZus{}bin3}\PY{o}{+}\PY{l+m+mi}{1}\PY{p}{)}\PY{o}{.}\PY{n}{cumprod}\PY{p}{(}\PY{p}{)}\PY{o}{\PYZhy{}}\PY{l+m+mi}{1}\PY{p}{)}

\PY{c+c1}{\PYZsh{} Plotting cumulative returns}
\PY{n}{plt}\PY{o}{.}\PY{n}{style}\PY{o}{.}\PY{n}{use}\PY{p}{(}\PY{l+s+s1}{\PYZsq{}}\PY{l+s+s1}{ggplot}\PY{l+s+s1}{\PYZsq{}}\PY{p}{)}
\PY{n}{plt}\PY{o}{.}\PY{n}{plot}\PY{p}{(}\PY{n}{np}\PY{o}{.}\PY{n}{array}\PY{p}{(}\PY{n}{cumulative\PYZus{}ret\PYZus{}bin1}\PY{p}{)}\PY{p}{,} \PY{n}{label}\PY{o}{=}\PY{l+s+s1}{\PYZsq{}}\PY{l+s+s1}{bin1}\PY{l+s+s1}{\PYZsq{}}\PY{p}{)}
\PY{n}{plt}\PY{o}{.}\PY{n}{plot}\PY{p}{(}\PY{n}{np}\PY{o}{.}\PY{n}{array}\PY{p}{(}\PY{n}{cumulative\PYZus{}ret\PYZus{}bin3}\PY{p}{)}\PY{p}{,} \PY{n}{label}\PY{o}{=}\PY{l+s+s1}{\PYZsq{}}\PY{l+s+s1}{bin3}\PY{l+s+s1}{\PYZsq{}}\PY{p}{)}
\PY{n}{plt}\PY{o}{.}\PY{n}{ylabel}\PY{p}{(}\PY{l+s+s1}{\PYZsq{}}\PY{l+s+s1}{cumulative returns}\PY{l+s+s1}{\PYZsq{}}\PY{p}{,} \PY{n}{fontsize}\PY{o}{=}\PY{l+m+mi}{22}\PY{p}{)} 
\PY{n}{plt}\PY{o}{.}\PY{n}{title}\PY{p}{(}\PY{l+s+s1}{\PYZsq{}}\PY{l+s+s1}{CaC40 index}\PY{l+s+s1}{\PYZsq{}}\PY{p}{)}
\PY{n}{plt}\PY{o}{.}\PY{n}{xticks}\PY{p}{(}\PY{p}{[}\PY{p}{]}\PY{p}{)}
\PY{n}{plt}\PY{o}{.}\PY{n}{legend}\PY{p}{(}\PY{p}{)}
\PY{n}{plt}\PY{o}{.}\PY{n}{show}\PY{p}{(}\PY{p}{)}
\end{Verbatim}
\end{tcolorbox}

    \begin{center}
    \adjustimage{max size={0.9\linewidth}{0.9\paperheight}}{output_38_0.png}
    \end{center}
    { \hspace*{\fill} \\}
    
    It appears that portfolio bin1 has higher return than portfolio bin2.

    \hypertarget{comparing-tda-portfolios-with-the-cac40-index}{%
\section{Comparing TDA portfolios with the CaC40
index}\label{comparing-tda-portfolios-with-the-cac40-index}}

The question to be ask after our TDA analysis is the following: Did we
finally beat the CaC40 index with our approach?

To answer to this question, we will represent on a single plot the
cummulative returns of the three portfolios: bin1, bin3 and the CaC40
index.

    \begin{tcolorbox}[breakable, size=fbox, boxrule=1pt, pad at break*=1mm,colback=cellbackground, colframe=cellborder]
%\prompt{In}{incolor}{87}{\boxspacing}
\begin{Verbatim}[commandchars=\\\{\}]
\PY{n}{cumulative\PYZus{}ret\PYZus{}index}\PY{o}{=}\PY{p}{(}\PY{p}{(}\PY{n}{asset\PYZus{}returns}\PY{p}{[}\PY{l+s+s1}{\PYZsq{}}\PY{l+s+s1}{\PYZca{}FCHI}\PY{l+s+s1}{\PYZsq{}}\PY{p}{]}\PY{o}{+}\PY{l+m+mi}{1}\PY{p}{)}\PY{o}{.}\PY{n}{cumprod}\PY{p}{(}\PY{p}{)}\PY{o}{\PYZhy{}}\PY{l+m+mi}{1}\PY{p}{)}

\PY{c+c1}{\PYZsh{} Plotting cumulative returns}
\PY{n}{plt}\PY{o}{.}\PY{n}{style}\PY{o}{.}\PY{n}{use}\PY{p}{(}\PY{l+s+s1}{\PYZsq{}}\PY{l+s+s1}{ggplot}\PY{l+s+s1}{\PYZsq{}}\PY{p}{)}
\PY{n}{plt}\PY{o}{.}\PY{n}{plot}\PY{p}{(}\PY{n}{np}\PY{o}{.}\PY{n}{array}\PY{p}{(}\PY{n}{cumulative\PYZus{}ret\PYZus{}bin1}\PY{p}{)}\PY{p}{,} \PY{n}{label}\PY{o}{=}\PY{l+s+s1}{\PYZsq{}}\PY{l+s+s1}{bin1}\PY{l+s+s1}{\PYZsq{}}\PY{p}{)}
\PY{n}{plt}\PY{o}{.}\PY{n}{plot}\PY{p}{(}\PY{n}{np}\PY{o}{.}\PY{n}{array}\PY{p}{(}\PY{n}{cumulative\PYZus{}ret\PYZus{}bin3}\PY{p}{)}\PY{p}{,} \PY{n}{label}\PY{o}{=}\PY{l+s+s1}{\PYZsq{}}\PY{l+s+s1}{bin3}\PY{l+s+s1}{\PYZsq{}}\PY{p}{)}
\PY{n}{plt}\PY{o}{.}\PY{n}{plot}\PY{p}{(}\PY{n}{np}\PY{o}{.}\PY{n}{array}\PY{p}{(}\PY{n}{cumulative\PYZus{}ret\PYZus{}index}\PY{p}{)}\PY{p}{,} \PY{n}{label}\PY{o}{=}\PY{l+s+s1}{\PYZsq{}}\PY{l+s+s1}{cac40 index}\PY{l+s+s1}{\PYZsq{}}\PY{p}{)}
\PY{n}{plt}\PY{o}{.}\PY{n}{ylabel}\PY{p}{(}\PY{l+s+s1}{\PYZsq{}}\PY{l+s+s1}{cumulative returns}\PY{l+s+s1}{\PYZsq{}}\PY{p}{,} \PY{n}{fontsize}\PY{o}{=}\PY{l+m+mi}{22}\PY{p}{)} 
\PY{n}{plt}\PY{o}{.}\PY{n}{title}\PY{p}{(}\PY{l+s+s1}{\PYZsq{}}\PY{l+s+s1}{Comparison between three portfolio returns}\PY{l+s+s1}{\PYZsq{}} \PY{p}{)}
\PY{n}{plt}\PY{o}{.}\PY{n}{xticks}\PY{p}{(}\PY{p}{[}\PY{p}{]}\PY{p}{)}
\PY{n}{plt}\PY{o}{.}\PY{n}{legend}\PY{p}{(}\PY{p}{)}
\PY{n}{plt}\PY{o}{.}\PY{n}{show}\PY{p}{(}\PY{p}{)}
\end{Verbatim}
\end{tcolorbox}

    \begin{center}
    \adjustimage{max size={0.9\linewidth}{0.9\paperheight}}{output_41_0.png}
    \end{center}
    { \hspace*{\fill} \\}
    
    This plot above shows that the portfolios bin1 and bin3 beat the
benchmark index by generating higher profit.

\hypertarget{conclusion}{%
\section{Conclusion}\label{conclusion}}

In this report, we used topological methods to propose an investment
strategy on the CaC40 index. We have extracted from the CaC40 index two
sub-portfolios which have better returns than the underlying benchmark
index. More precisely, instead of investing on the whole CaC40 index, we
would recommend the following investment strategy according to you risk
appetite:

\begin{itemize}
\tightlist
\item
  If your have a high-risk profile, then you should invest (under the
  above sugested proportions) on bin1: Soci\'{e}t\'{e} G\'{e}n\'{e}rale, Michelin,
  Engie, Teleperformance, Bouygues, BNP Paribas, Veolia Environ,
  Carrefour, Kering, and finally Schneider Electric.
\end{itemize}

As shown in this analysis, the risk is relatively high on this portfolio
but the return is be much higher than that of other portfolios including
the benchmark index.

\begin{itemize}
\tightlist
\item
  If you have a low-risk profile, then you should invest (under the
  above sugested proportions) on bin3: Pernod Ricard, Atos, Orange,
  Airbus, Accor, Peugeot, Sodexo, Saint-Gobain, Cr\'{e}dit Agricole and
  finally Thales.
\end{itemize}

This latter portfolio is less risky than bin1. The return is not that
high but it is still benefict to invest on this one than the benchmark
index.

Note that our recommendations do not take into account the current
situation and a wise investor should also track the financial (S1,S2 and
annual) reports of these companies.

    \hypertarget{declaration-of-competing-interest}{%
\section{Declaration of Competing
Interest}\label{declaration-of-competing-interest}}

The author declares that he has no known competing financial interests
or personal relationships that could have appeared to influence the work
reported in this tutorial.

    \hypertarget{perspectives}{%
\section{Perspectives}\label{perspectives}}

What else can we do after this investigation?

\begin{itemize}
\item
  It is possible to compare the performance of our strategy with the
  traditional enhanced indexing methods such as Excess Mean Return. A
  research in this direction was done in {[}3{]};
\item
  This approach can be used on other stock indexes and any existing
  portfolio. If you are a portfolio manager, then you can empirically
  compare our approach with your own strategy and then broaden you
  management tools.
\item
  If you think the returns we got in this analysis were not high enough,
  then you can tun the meta parameters such as d (taken embedding
  dimension) and p (for the \(L^p\) norm) and all others. By this
  process, you could increase the profit.
\end{itemize}

    \hypertarget{references}{%
\section{References}\label{references}}

    {[}1{]} Peter Bubenik. Statistical topological data analysis using
persistence land- scapes. The Journal of Machine Learning Research,
16(1):77--102, 2015.

{[}2{]} Marian Gidea and Yuri Katz. Topological data analysis of
financial time series: Landscapes of crashes. Physica A: Statistical
Mechanics and its Ap- plications, 491:820--834, 2018.

{[}3{]} Anubha Goel, Puneet Pasricha, and Aparna Mehra. Topological data
analysis in investment decisions. Expert Systems with Applications,
147:113222, 2020.


    % Add a bibliography block to the postdoc
    
    
    
\end{document}
